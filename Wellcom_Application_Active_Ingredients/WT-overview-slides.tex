% Options for packages loaded elsewhere
\PassOptionsToPackage{unicode}{hyperref}
\PassOptionsToPackage{hyphens}{url}
%
\documentclass[
  ignorenonframetext,
]{beamer}
\usepackage{pgfpages}
\setbeamertemplate{caption}[numbered]
\setbeamertemplate{caption label separator}{: }
\setbeamercolor{caption name}{fg=normal text.fg}
\beamertemplatenavigationsymbolsempty
% Prevent slide breaks in the middle of a paragraph
\widowpenalties 1 10000
\raggedbottom
\setbeamertemplate{part page}{
  \centering
  \begin{beamercolorbox}[sep=16pt,center]{part title}
    \usebeamerfont{part title}\insertpart\par
  \end{beamercolorbox}
}
\setbeamertemplate{section page}{
  \centering
  \begin{beamercolorbox}[sep=12pt,center]{part title}
    \usebeamerfont{section title}\insertsection\par
  \end{beamercolorbox}
}
\setbeamertemplate{subsection page}{
  \centering
  \begin{beamercolorbox}[sep=8pt,center]{part title}
    \usebeamerfont{subsection title}\insertsubsection\par
  \end{beamercolorbox}
}
\AtBeginPart{
  \frame{\partpage}
}
\AtBeginSection{
  \ifbibliography
  \else
    \frame{\sectionpage}
  \fi
}
\AtBeginSubsection{
  \frame{\subsectionpage}
}
\usepackage{lmodern}
\usepackage{amssymb,amsmath}
\usepackage{ifxetex,ifluatex}
\ifnum 0\ifxetex 1\fi\ifluatex 1\fi=0 % if pdftex
  \usepackage[T1]{fontenc}
  \usepackage[utf8]{inputenc}
  \usepackage{textcomp} % provide euro and other symbols
\else % if luatex or xetex
  \usepackage{unicode-math}
  \defaultfontfeatures{Scale=MatchLowercase}
  \defaultfontfeatures[\rmfamily]{Ligatures=TeX,Scale=1}
\fi
% Use upquote if available, for straight quotes in verbatim environments
\IfFileExists{upquote.sty}{\usepackage{upquote}}{}
\IfFileExists{microtype.sty}{% use microtype if available
  \usepackage[]{microtype}
  \UseMicrotypeSet[protrusion]{basicmath} % disable protrusion for tt fonts
}{}
\makeatletter
\@ifundefined{KOMAClassName}{% if non-KOMA class
  \IfFileExists{parskip.sty}{%
    \usepackage{parskip}
  }{% else
    \setlength{\parindent}{0pt}
    \setlength{\parskip}{6pt plus 2pt minus 1pt}}
}{% if KOMA class
  \KOMAoptions{parskip=half}}
\makeatother
\usepackage{xcolor}
\IfFileExists{xurl.sty}{\usepackage{xurl}}{} % add URL line breaks if available
\IfFileExists{bookmark.sty}{\usepackage{bookmark}}{\usepackage{hyperref}}
\hypersetup{
  pdftitle={Wellcome Trust Application Overview},
  pdfauthor={Argyris Stringaris},
  hidelinks,
  pdfcreator={LaTeX via pandoc}}
\urlstyle{same} % disable monospaced font for URLs
\newif\ifbibliography
\setlength{\emergencystretch}{3em} % prevent overfull lines
\providecommand{\tightlist}{%
  \setlength{\itemsep}{0pt}\setlength{\parskip}{0pt}}
\setcounter{secnumdepth}{-\maxdimen} % remove section numbering

\title{Wellcome Trust Application Overview}
\author{Argyris Stringaris}
\date{04/06/2022}

\begin{document}
\frame{\titlepage}

\begin{frame}{Main Goal of the Application.}
\protect\hypertarget{main-goal-of-the-application.}{}

Discover what causes improvement in successful psychological therapy for
anxiety and depression in youth.

\end{frame}

\begin{frame}{Principal Hypothesis.}
\protect\hypertarget{principal-hypothesis.}{}

Improvement in psychotherapy depends on the generation of surprises that
are salient to the individual and their position in the world.

We call this the salient \textbf{surprise hypothesis (SSH)}.

We posit that it is fundamental to success in psychotherapy

\end{frame}

\begin{frame}{Disorders studied}
\protect\hypertarget{disorders-studied}{}

To achieve this, we will :

\begin{enumerate}
[a)]
\item
  demonstrate that the \textbf{SSH} is key to the success of cognitive
  therapy in social anxiety, one of the most efficacious and best-
  studied psychotherapies.
\item
  show that the \textbf{SSH} also underlies improvement of depression in
  those with social phobia comorbid with depression, a common and
  debilitating comorbdiity.
\item
  examine how SSH can be used to improve treatment in depression not
  comorbid with social anxiety.
\end{enumerate}

\end{frame}

\begin{frame}{Main Research Approaches}
\protect\hypertarget{main-research-approaches}{}

Investigations will be conducted at two levels in an iterative way:

\begin{enumerate}
\item
  \emph{Experimental Medicine Design}: Single-Session Psychological
  Therapy Laboratory to achieve causal manipulation and tight
  experimental control of variables.
\item
  \emph{Randomised Controlled Trial}: Clinical trial that allows us to
  test mediating mechanisms in actual treatment.
\end{enumerate}

Both will be co-researched together with young people\ldots.ADD HERE

\end{frame}

\begin{frame}{Methodologies involved}
\protect\hypertarget{methodologies-involved}{}

\begin{itemize}
\tightlist
\item
  neural (MEG)
\item
  cognitive
\item
  eye tracking
\item
  physiology
\item
  interoception
\item
  video taping and analysis of images
\item
  high-density data collection
\end{itemize}

\end{frame}

\begin{frame}{Clinical Rationale}
\protect\hypertarget{clinical-rationale}{}

\begin{itemize}
\tightlist
\item
  SAD and Depression two of the most common and debilitating disorders,
  frequently comorbid.
\item
  SAD greatly treatable. Deression in great need of improvement.
\end{itemize}

\end{frame}

\begin{frame}{Research Rationale}
\protect\hypertarget{research-rationale}{}

\begin{itemize}
\tightlist
\item
  Strong theory about how therapy works.
\item
  Lots of background mediation analyses, experimental designs.
\item
  Paves the way for getting as clause to causality as possible.
\end{itemize}

\end{frame}

\begin{frame}{What are salient surprises?}
\protect\hypertarget{what-are-salient-surprises}{}

\begin{itemize}
\item
  Describe what would constitute a surprise in psychotherapy,
  i.e.~outcome better than expected
\item
  Describe why generalisation is important.
\end{itemize}

A central Aim here will be to define, operationalise, model and
manipulate those surprises.

\end{frame}

\begin{frame}{How are surprises generated in SAD?}
\protect\hypertarget{how-are-surprises-generated-in-sad}{}

\begin{itemize}
\item
  In CT for SAD, some of the principal ways are: dropping of safety
  behaviours and shifting SFA
\item
  Experimental evidence. Here, we will focus in on manipulating etc.
\end{itemize}

\end{frame}

\begin{frame}{How are surprises generated in Depression?}
\protect\hypertarget{how-are-surprises-generated-in-depression}{}

\begin{itemize}
\item
  Not clear, several things to examine
\item
  Avoidance one principal mechanism.
\end{itemize}

\end{frame}

\begin{frame}{First phase: Identify mechanisms of change in SAD}
\protect\hypertarget{first-phase-identify-mechanisms-of-change-in-sad}{}

\begin{itemize}
\item
  Experimental session recreate previous manipulation of SFA/Safety Beh
  (multiple events!!!)
\item
  Measure SFΑ, Safety Beh (quantify in various ways, measure neural
  correlates) and manipulate
\item
  How can we quantify this in the clinical sample? Will need measures of
  these that are easy to get during treatment.
\item
  Observational: surprises mediate between SFA/SB Anxiety
\end{itemize}

\end{frame}

\begin{frame}{First phase: Identify mechanisms of change in SAD}
\protect\hypertarget{first-phase-identify-mechanisms-of-change-in-sad-1}{}

\begin{itemize}
\item
  Experimental I: is there a way of tweaking expectation or outcome in
  an ethically viable way? For example: generate zero surprises or
  negative ones, despite dropping safety behaviours?
\item
  Experimental II: how to tweak the generalisability of surprise?
  Through instructions? Would simply dropping the instructions about
  ``what does this mean about the world generally'' be sufficient?
\end{itemize}

\end{frame}

\begin{frame}{Second phase: Manipulate surprises in SAD comorbid with
depression}
\protect\hypertarget{second-phase-manipulate-surprises-in-sad-comorbid-with-depression}{}

\begin{itemize}
\item
  It is the same as above, with two crucial additions.
\item
  First, measure mood throughout using multivariate measures:
  questionnaires, physiology, face processing etc.
\item
  Second, test directions of effect and possible reciprocal
  relationships. Do manipulations such as SFA refocusing change mood in
  their own right? Does this facilitate changes in anxiety or the
  generation of surprises? Does low mood impact on generalisation?
\end{itemize}

\end{frame}

\begin{frame}{Second phase: Manipulate surprises in SAD comorbid with
depression}
\protect\hypertarget{second-phase-manipulate-surprises-in-sad-comorbid-with-depression-1}{}

\begin{itemize}
\item
  Third, test mechanisms that may be specific to depression, e.g.~access
  to autobiographical memory during session/experiment that may (a)
  lower mood; (b) discount outcomes and therefore surprises
\item
  Test these with tighter experimental control and measurement in the
  clinical sample. Is there a lag between depression and anxiety that we
  can exploit for mediation analyses? To do within and between session.
\end{itemize}

\end{frame}

\begin{frame}{Third phase: Test the above in depression not comorbid
with SAD}
\protect\hypertarget{third-phase-test-the-above-in-depression-not-comorbid-with-sad}{}

\begin{itemize}
\tightlist
\item
  Apply the above to depression not comorbid with SAD.
\end{itemize}

\end{frame}

\begin{frame}{What would a session for SAD look like?}
\protect\hypertarget{what-would-a-session-for-sad-look-like}{}

\begin{itemize}
\tightlist
\item
  It would be like in Eleanor's here: Leigh, E., Chiu, K., \& Clark, D.
  M. (2021). Self-focused attention and safety behaviours maintain
  social anxiety in adolescents: An experimental study. PloS one, 16(2),
  e0247703. \url{https://doi.org/10.1371/journal.pone.0247703} 
\end{itemize}

with two important modifications/additions:

\begin{itemize}
\item
  multi-trial \textbf{ask EL and GK about this}
\item
  multi-method assessment of mediators and outcomes, including attention
  and behaviours, but also mood.
\end{itemize}

\end{frame}

\begin{frame}{Questions (primarily) for Eleanor and Georgina}
\protect\hypertarget{questions-primarily-for-eleanor-and-georgina}{}

\begin{enumerate}
\item
  How can we turn the single-event session into a series of blocks or
  events (block vs event design, respectively). For example, can we
  instruct the participants to switch between internal vs external focus
  of attention several times per session? Are we too worried about carry
  over?
\item
  How can we best manipulate surprise generalisation. In the slides that
  I have sent you, I am referring to the Salient Surprise Hypothesis,
  which is a shorthand for a surprise that is meaningful in the
  individual's world. I have some ideas above about how we might do it.
\end{enumerate}

\end{frame}

\begin{frame}{Questions (primarily) for Eleanor and Georgina}
\protect\hypertarget{questions-primarily-for-eleanor-and-georgina-1}{}

\begin{enumerate}
\setcounter{enumi}{2}
\tightlist
\item
  Eleanor: is your work for MRC going to be an RCT? It will be good to
  think what measures we could use that are both innovative and
  easy/unintrusive to incorporate (e.g.~some form of eye-tracking).
\end{enumerate}

\end{frame}

\begin{frame}{Questions (primarily) for Elizabeth}
\protect\hypertarget{questions-primarily-for-elizabeth}{}

\begin{enumerate}
\tightlist
\item
  Do you agree that SFA and SB can be mediators of change in BA?
\item
  Do you agree with generalisation as the target in BA?
\item
  Can you think of experiments where this has been sufficiently shown?
\item
  Other thoughts re: BA and its possible differences or alternative
  explanations?
\end{enumerate}

\end{frame}

\begin{frame}{Questions for Quentin and Diego}
\protect\hypertarget{questions-for-quentin-and-diego}{}

Georgina, Eleanor you and I should think about how best to test the idea
of \emph{meaningful} shifts in beliefs. As discussed from a
computational perspective, I think that it would be good to distinguish
between mere surprise vs epistemic surprises. I found the Schwartenbeck
paper here: particularly informative in this regard. Schwartenbeck P,
FitzGerald THB, Dolan R.(2016) Neural signals encoding shifts in beliefs
NeuroImage 125, 578 Thoughts on this?

\begin{enumerate}
\tightlist
\item
  Do you think we could develop a task to differentiate the two, but is
  done as part of a session? I am thinking of a design where
  generalisation instructions are provided/withheld per trial, but need
  to think harder. Ideas?
\end{enumerate}

\end{frame}

\begin{frame}{Questions for Quentin and Diego}
\protect\hypertarget{questions-for-quentin-and-diego-1}{}

\begin{enumerate}
\setcounter{enumi}{1}
\item
  We should do this in MEG (though we would have to think about the
  depth of the dopaminergic strutures).
\item
  It is tempting to consider pharmacological manipulation of the
  dopamine signal-- Thoughts?
\end{enumerate}

\end{frame}

\end{document}
