% Options for packages loaded elsewhere
\PassOptionsToPackage{unicode}{hyperref}
\PassOptionsToPackage{hyphens}{url}
%
\documentclass[
]{article}
\usepackage{lmodern}
\usepackage{amssymb,amsmath}
\usepackage{ifxetex,ifluatex}
\ifnum 0\ifxetex 1\fi\ifluatex 1\fi=0 % if pdftex
  \usepackage[T1]{fontenc}
  \usepackage[utf8]{inputenc}
  \usepackage{textcomp} % provide euro and other symbols
\else % if luatex or xetex
  \usepackage{unicode-math}
  \defaultfontfeatures{Scale=MatchLowercase}
  \defaultfontfeatures[\rmfamily]{Ligatures=TeX,Scale=1}
\fi
% Use upquote if available, for straight quotes in verbatim environments
\IfFileExists{upquote.sty}{\usepackage{upquote}}{}
\IfFileExists{microtype.sty}{% use microtype if available
  \usepackage[]{microtype}
  \UseMicrotypeSet[protrusion]{basicmath} % disable protrusion for tt fonts
}{}
\makeatletter
\@ifundefined{KOMAClassName}{% if non-KOMA class
  \IfFileExists{parskip.sty}{%
    \usepackage{parskip}
  }{% else
    \setlength{\parindent}{0pt}
    \setlength{\parskip}{6pt plus 2pt minus 1pt}}
}{% if KOMA class
  \KOMAoptions{parskip=half}}
\makeatother
\usepackage{xcolor}
\IfFileExists{xurl.sty}{\usepackage{xurl}}{} % add URL line breaks if available
\IfFileExists{bookmark.sty}{\usepackage{bookmark}}{\usepackage{hyperref}}
\hypersetup{
  hidelinks,
  pdfcreator={LaTeX via pandoc}}
\urlstyle{same} % disable monospaced font for URLs
\usepackage[margin=1in]{geometry}
\usepackage{graphicx,grffile}
\makeatletter
\def\maxwidth{\ifdim\Gin@nat@width>\linewidth\linewidth\else\Gin@nat@width\fi}
\def\maxheight{\ifdim\Gin@nat@height>\textheight\textheight\else\Gin@nat@height\fi}
\makeatother
% Scale images if necessary, so that they will not overflow the page
% margins by default, and it is still possible to overwrite the defaults
% using explicit options in \includegraphics[width, height, ...]{}
\setkeys{Gin}{width=\maxwidth,height=\maxheight,keepaspectratio}
% Set default figure placement to htbp
\makeatletter
\def\fps@figure{htbp}
\makeatother
\setlength{\emergencystretch}{3em} % prevent overfull lines
\providecommand{\tightlist}{%
  \setlength{\itemsep}{0pt}\setlength{\parskip}{0pt}}
\setcounter{secnumdepth}{-\maxdimen} % remove section numbering

\author{}
\date{\vspace{-2.5em}}

\begin{document}

\hypertarget{curriculum-vitae-of-argyris-stringaris-md-phd-frcpsych}{%
\subsection{Curriculum Vitae of Argyris Stringaris, MD, PhD,
FRCPsych}\label{curriculum-vitae-of-argyris-stringaris-md-phd-frcpsych}}

\textbf{Professor and Chair of Child \& Adolescent Psychiatry at
University College London}

\includegraphics{https://github.com/Argyris36/CV/blob/gh-pages/75246595_736833856820733_2719593851835645952_n-2.jpg}

email:
\href{mailto:argyris.stringaris@nih.gov}{\nolinkurl{argyris.stringaris@nih.gov}}\\
Address for Correspondence: 1-19 Torrington Pl, London WC1E 7HB, United
Kingdom

Date: May 2022

\hypertarget{overview}{%
\subsubsection{Overview}\label{overview}}

I am a clinician and a neuroscientist. I was appointed Chair of Child
and Adolescent Psychiatry at UCL in January 2022. Until then, I was
Senior Investigator and Chief of the Section of Clinical and
Computational Psychiatry at NIMH/NIH in the USA and before that a Senior
Lecturer and a Wellcome Trust Fellow at the Institute of Psychiatry
Psychology and Neuroscience, King's College London. I trained in Child
and Adolescent Psychiatry at the Maudsley Hospital.

I have two main research aims.

In terms of basic research, my aim is to understand how affective
phenomena (variably termed moods, emotions, feelings or affects) are
generated and maintained.

In terms of clinical research, I study interventions that reduce the
negative impact that affective phenomena, particularly depression, have
on young people and families.

I am 47 years old, a husband and father of three daughters. I row
competitively on Concept2 and read hard-copy books.

Here is a
\href{https://pubmed.ncbi.nlm.nih.gov/?term=stringaris+a\&sort=date}{list
of my publications} and here is my
\href{https://scholar.google.com/citations?user=9B82424AAAAJ\&hl=en\&oi=ao}{Google
Scholar page}.

\hypertarget{current-positions}{%
\subsubsection{Current Positions}\label{current-positions}}

\textbf{Professor of Child and Adolescent Psychiatry} Department of
Psychiatry and Department of Clinical, Educational \& Health Psychology
University College London

\hypertarget{other-positions}{%
\subsubsection{Other positions}\label{other-positions}}

\textbf{Professor of Child and Adolescent Psychiatry}\\
University of Athens, Greece

\textbf{Visiting Scientist and Head of the Clinical Neuroscience of Mood
Disorders in Children and Adolescents}\\
Central Institute of Mental Health (Zentralinstitut für Seelische
Gesundheit), Mannheim, Germany

\textbf{Honorary Consultant Child \& Adolescent Psychiatrist} Camden \&
Islington NHS Foundation Trust

\hypertarget{degrees-and-training}{%
\subsubsection{Degrees and Training}\label{degrees-and-training}}

\textbf{Medical Training and Qualifications/Licenses to Practice}

\begin{itemize}
\item
  May 2000 MBBS (MD) University of Göttingen, Germany. Top 3\% of all
  German medical students in the 5th Year medical final examinations (II
  Staatsexamen) and top 20\% of all students in 6th Year examination
  (III Staatsexamen).
\item
  Registration with the UK General Medical Council: 6043066
\item
  Registration with the Maryland Board of Physicians by ``Conceded
  Eminence'': D48427
\item
  Entry into the UK Specialist Register for Child and Adolescent
  Psychiatry (CCT): 2011
\item
  Royal College of Psychiatrists Membership: 813569, Since 2018 a Fellow
  of the College.
\item
  Formerly Section 12 approved (lapsed since in the USA)
\item
  Hellenic Medical Council: Certified Specialist in Child and Adolescent
  Psychiatry
\end{itemize}

\textbf{Research Training}

\begin{itemize}
\item
  PhD: King's College London, University of London, UK, May 2011\\
\item
  Research Fellow, Bipolar Spectrum Disorders, NIMH, USA, Jul 2008 --
  Jul 2009
\item
  Dr med (MD Research), Department of Neurology, University of
  Göttingen, Germany, May 2000
\end{itemize}

\textbf{Training in Neurology and Psychiatry}

\begin{itemize}
\tightlist
\item
  Aug 2006 -- Apr 2011: Higher Specialist Training in Child and
  Adolescent Psychiatry, Maudsley Hospital and Great Ormond Street and
  Royal London Rotation\\
\item
  Jul 2006: Membership, Royal College of Psychiatrists (MRCPsych,
  London, UK)\\
\item
  Oct 2003 - Aug 2006: Senior House Officer, Maudsley Hospital, London,
  UK\\
\item
  Oct 2002 - Oct 2003: Senior House Officer, Maudsley Hospital, London,
  UK and Institute of Psychiatry, London, UK\\
\item
  Jun 2000 - Oct 2002: Neurology and Medicine, University of Göttingen,
  Germany
\end{itemize}

\hypertarget{employment}{%
\subsubsection{Employment}\label{employment}}

\begin{itemize}
\tightlist
\item
  August 2016 to 2021: Chief of Section of Clinical \& Computational
  Psychiatry, NIMH, NIH
\item
  Jan 2012 - August 2016 Senior Lecturer (Tenured University Position)
  Head of Mood and Development Laboratory Consultant Child and
  Adolescent Psychiatrist (Attending Physician) Department of Child \&
  Adolescent Psychiatry, Institute of Psychiatry, Psychology \&
  Neuroscience, King's College London \& Maudsley Hospital London
\item
  Apr 2011 -- August 2016: Consultant Psychiatrist at Maudsley Hospital
\item
  Jan 2007 - Jan 2012: Lecturer in Child and Adolescent Psychiatry at
  Institute of Psychiatry, King's College London, London, UK with an
  interim period as Fellow at the National Institute of Mental Health,
  Bethesda, USA between July 2008 and July 2009.
\item
  Jan 2007 - Apr 2011: Specialist Registrar, Maudsley Hospital, London,
  UK
\item
  Aug 2006 - Dec 2006: Honorary Specialist Registrar in Child and
  Adolescent Psychiatry, Great Ormond Street and Royal London Rotation,
  London, UK
\item
  Apr 2006 -- Aug 2006: Maudsley and King's College Hospitals, General
  Liaison and Perinatal Psychiatry, London, UK
\item
  Oct 2002 - Aug 2006: Senior House Officer, South London and Maudsley
  NHS General Psychiatry Training Rotation, London, UK
\item
  Jun 2000 - Oct 2002: Senior House Officer (Assistenzarzt), Department
  of Neurology, University of Göttingen, Germany
\end{itemize}

\hypertarget{scientific-leadership}{%
\subsubsection{Scientific Leadership}\label{scientific-leadership}}

\textbf{Presidency of International Organization}

\begin{itemize}
\tightlist
\item
  President of the
  \href{https://isrcap.org/executive-committee.html}{International
  Society for Research in Child and Adolescent Psychopathology (ISRCAP)}
\end{itemize}

\textbf{Journal Editorships}

\begin{itemize}
\tightlist
\item
  Editor European Journal of Child and Adolescent Psychiatry (2021 -
  ongoing)
\item
  Consulting Editor Cognitive Affective and Behavioral Neuroscience
  (2021 - ongoing)
\item
  Editor (2012-2019) Journal of Child Psychology and Psychiatry
\item
  Editorial Board (2015 ongoing): Journal of the American Academy of
  Child \& Adolescent Psychiatry
\end{itemize}

\textbf{Taskforce and Scientific Board Memberships} I have been invited
to participate as a member or scientific advisor in the following: -
Member of the European College of Neuropharmacology (ECNP) Child and
Adolescent Psychiatry Network. - Member Scientific Advisory Group of
Improving Adolescent mentaL health by reducing the Impact of PoVErty
(ALIVE), a Wellcome trust-funded international project. - Scientific
Board European Society of Child and Adolescent Psychiatry (ESCAP) -
Member of European ADHD Guidelines Group (EAGG) - Member of the American
Academy of Child and Adolescent Psychiatry Presidential Initiative Task
Force on Emotion Regulation in Children : ``Coming Together to Treat the
Sickest Kids''- I was appointed chair of the Measurement Subcommittee -
Member of the Program Committee for the American Academy of Child and
Adolescent Psychiatry (AACAP) Annual meeting 2021 - Scientific Advisory
Board MQ The Mental Health Charity, UK - Member of the Task Force on
Child and Youth Psychiatry of the World Federation of Societies of
Biological Psychiatry (WFSBP) - Scientific Advisory Board Advisory Board
for the Medical Research Council UK initiative in Adolescence, Mental
Health and the Developing Mind

\hypertarget{prizes}{%
\subsubsection{Prizes}\label{prizes}}

\begin{itemize}
\tightlist
\item
  2021 Open Neuro Hall of Fame, tied 2nd Place for contributing datasets
\item
  2021 Kramer Pollnow Prize Award for scientific excellence in clinical
  research in\\
  child,adolescent and adult psychiatry by the European Network of
  Hyperactivity Disorders (EUNETHYDIS)
\item
  2020 Elaine Schlosser Lewis Award as the best research paper in ADHD
  research this year for the paper A Double-Blind Randomized
  Placebo-Controlled Trial of Citalopram Adjunctive to Stimulant
  Medication in Youth With Chronic Severe Irritability (first authors:
  Dr Kenneth Towbin, Dr Pablo Vidal-Ribas, see under Publications) that
  I am a last author on.
\item
  2019 NIH Director's Award ``for exemplary performance while
  demonstrating significant leadership, skill and ability in serving as
  a mentor.''\\
\item
  2019 Gerald L Klerman Young Investigator (under 45 years) Prize, the
  highest honor that the Depression and Bipolar Support Alliance gives
  to members of the scientific community
\item
  2019 Prize for Distinguished Editorial Contributions, Academy of Child
  and Adolescent Mental Health, J Child Psychol and Psychiatry
\item
  2018 Elected a Fellow of the Royal College of Psychiatrists UK for
  Distinct and Significant Contributions to psychiatry.
\item
  2018 Outstanding Mentor Award by the National Institutes of Mental
  Health
\item
  2016 Special Commendation by the British Medical Association (BMA) for
  our book Disruptive Mood: Irritability in Children and Adolescents,
  published by Oxford University Press
\item
  2014 Best Paper of the Year in Depression or Suicide published in
  JAACAP, awarded by the Klingenstein Foundation
\item
  2010 Research Prize, European Psychiatric Association (EPA)\\
\item
  2004 Special Mention from the International Neuropsychiatry
  Association
\item
  2004 European Science Foundation Young Scientist Travel Award
\end{itemize}

\hypertarget{keynote-lectures-invited-lectures-symposia-and-chairing-of-symposia-examples}{%
\subsubsection{Keynote Lectures, Invited Lectures, Symposia and Chairing
of Symposia
(examples)}\label{keynote-lectures-invited-lectures-symposia-and-chairing-of-symposia-examples}}

-Chair: ADHD. Life-time continuity, serious mental illness, and genetics
World Psychiatric Association Thematic Conference, Athens, Greece -
Guest Speaker and Panelist Pediatric Major Depressive Disorder
Mini-Symposium, Co-hosted by the Division of Pediatric and Maternal
Health and the Division of Psychiatry Office of New Drugs \textbar{}
Center for Drug Evaluation \& Research \textbar{} U.S. Food and Drug
Administration (FDA), (June 2022) Bethesda, MD, USA - Keynote Lecture
(Lettura Magistrale) at the joint meeting of the Italian Societies of
Neuropsychopharmacology and Child \& Adolescent Psychiatry (May 2022),
Cagliari, Italy, Title: What is mood and how to modify it - Keynote
Lecture (Hauptvortrag) at the German Society for Child and Adolescent
Psychiatry (Deutsche Gesellschaft für Kinder un Jugendpsychiatrie,
DGKJP, May 2022), Magdeburg Germany, Title: Warum gibt es Depression -
Annual Distinguished Scientist Lecture at Pittsburgh University (April
2022) The Why and How of Mood: Theoretical and Computational Approaches
for Clinicians and Researchers in Depression. - Opening Lecture at the
European College of Neuropsychopharmacology (ENCP) at\\
Servolo,Venice (March 2022) Challenges and New Directions in Affective
Nosology - Grand Rounds at Cornell University Department of Psychiatry
(February 2022) What is mood? A conceptual and computational account of
some basic questions about depression. - Chair Symposium Irritability
and Reactive Aggression: Implications for Diagnosis, Treatment, and
Equity, American Academy of Child and Adolescent Psychiatry, October
2021 - Maryland University Grand Rounds, May 2021 - New York University,
Langone, Grand Rounds May 2021 - European Society for Child and
Adolescent Psychiatry (ESCAP) Expert Day, June 2021 - Royal College of
Psychiatrists, Talk at Symposium S36 Dimensional and Categorical
Psychopathology in Youth, Talk: Research in Depression, Great
Expectations and Great Challenges, June 2021 - Society for Biological
Psychiatry, Talk at Symposium on Developmental Computational Psychiatry,
Title: Adolescent Mood Dynamics, May 2021 - Society for Biological
Psychiatry, Talk at Symposium on Emerging Tools and Technologies, Title:
What is Mood: In Search of Models of Mood Across Health and Disease, May
2021 - Chair of the Symposium on Intergenerational Transmission of
Psychopathology and Early Identification of Risk: New Insights From the
Study of Child and Adolescent Offspring of Parents Living With
Depression, Bipolar Disorder, and Schizophrenia, at the Academy of Child
and Adolescent Psychiatry, San Francisco, USA, October 2020 - Invited
Speaker at University College London Institute of Mental Health First
International Conference - Invited Speaker Canadian Academy of Child and
Adolescent Psychiatry, Calgary, Canada, September 2020 (to be delivered
online). - Invited Speaker, Dalhousie University, Halifax, Nova Scotia,
Canada, September 2020 (to be delivered online). Panelist, Panel on
Innovative Strategies for Transforming Treatment of Depression and
Self-Harm, National Institutes of Health, September 2019 - Invited
Symposium Chair at the American Academy of Child and Adolescent
Psychiatry (``Novel Approaches to Inform Treatment Decisions in Child
Psychiatry: Steps Toward Personalized Medicine'') October 2019 - Invited
Lecture at Washington University in St Louis, Developmental Neuroimaging
group, April 2019 - Invited Lecture at the British Neuroscience
Association (BNA) Annual Conference in Dublin, April 2019 - Invited
Lecture Oxford University's Wolfson College at a workshop on
experimental medicine of Treatment Resistant Depression, June 2019 -
Invited Keynote Lecture, Royal College of Psychiatrists, Child and
Adolescent Psychiatry Annual Conference, Glasgow UK, September 2018 -
Invited Keynote Lecture, Scientific Society of Autism Spectrum Disorders
(WGAS), Annual Conference, Augsburg, Germany, February 2019 - Grand
Rounds (Weller Memorial Lecture), University of Pennsylvania,
Philadelphia, June 2017 - Invited Lecture, Laureate Institute, Tulsa,
OH, August 2017 - Grand Rounds at Georgetown University, Washington DC,
July 2017 - Grand Rounds at George Washington University, Washington DC,
October 2016 - Invited Chairing of Session, Biannual Irritability
Meeting, University of Vermont, Vermont, October 2017 - Plenary Lecture,
Stockholm University, Stockholm, Sweden, September 2016 - Invited
Lecture, Oxford University, Oxford, UK, January 2014 - Invited Lecture,
University of Vermont, Biannual Irritability Meeting, Vermont, October
2015 - Annual Meeting of the Royal College Lecture, Brighton, UK,
September 2015 - Invited Lecture, Child Psychiatry Research Society, UK,
July 2015, - Invited Lecture, Judge Baker Children's Center, Harvard
University, USA February 2111 - Invited Lecture, Harvard Dept
Psychology, February 2011

\hypertarget{grants}{%
\subsubsection{Grants}\label{grants}}

\begin{itemize}
\item
  Title: Regulating Emotions - Strengthening Adolescent Resilience
  (RE-STAR, Identifying Number: MR/W002493/1:)\\
  Position: Co-PI\\
  Funding Body: UK Research and Innovation (UKRI)\\
  Amount: £ 3,3 Million\\
  Dates: 2021 to FY 2025
\item
  Title: Characterization and Treatment of Adolescent Depression
  (Identifying Number: ZIA MH002957-03)\\
  Position: Principal Investigator\\
  Funding Body: National Institutes of Mental Health\\
  Amount: \$ 12,7 Million Dates: FY18 to FY 22
\item
  Title: Scoping review to systematically map the literature relating to
  the effects of SSRI treatment in young people aged 14-24 years old
  with depression and/or anxiety disorders.\\
  Position: Co-PI\\
  PI: Professor Catherine Harmer, Oxford University\\
  Funding Body: Wellcome Trust, UK\\
  Amount £ 45,000\\
  Dates: FY 2020 to FY 2021
\item
  Title: Dynamics of mood fluctuations and brain connectivity in
  adolescent depression\\
  Position Consultant, Grant number: 2019-175 PI: Dr Mattilde Vaghi
  Funding Body: Brain \& Behavior Research Foundation Amount: \$70,000\\
  Dates: FY 2020 to FY 2021
\item
  Title: Ketamine for severe adolescent depression: intermediate-term
  safety and efficacy\\
  Position: Consultant\\
  PI: Dr Jennifer Dwyer, Yale University\\
  Funding Body: Klingenstein Foundation\\
  Amount: \$60,000\\
  Dates: 5/15/2019 - 5/14/2021
\item
  Title: Me\_Health\_e: testing the added value of electronic outcome
  measurement in Child \& Adolescent Mental Health Services\\
  Position: PI\\
  Funding Body: Guys King's St Thomas's Charity\\
  Amount: £87,500\\
  Awarded: March 2016
\item
  Title: Validating the Case Register Interactive Search (CRIS) system
  for large naturalistic treatment trials in youth: A two-phase study
  using Attention Deficit Hyperactivity Disorder (ADHD) as a model.\\
  Position: Co-PI with (Goodman, R., Pickles, A., Simonoff, E)\\
  Funding Body: Guy's \& St Thomas' Charity\\
  Amount: £ 54,483.86 Dates: 2013 - 2014
\item
  Title: Ketamine's Actions on Rumination Mechanisms as an
  Antidepressant (KARMA) Position: Co-PI with Drs Mehta and Curran\\
  Funding Body: University College London and Johnson \& Johnson
  Innovation Awards\\
  Amount: £180,000\\
  Awarded: September 2015
\item
  Title: Wellcome Trust Enhancement Award for Brain effects of
  lurasidone in a double blind, randomized placebo-controlled study.\\
  Position CI\\
  Funding Body: Wellcome Trust\\
  Amount: £43,424\\
  Awarded: March 2014
\item
  Title: A double-blind, randomized, placebo-controlled study of
  single-dose lurasidone effects on regional cerebral blood flow,
  emotion- and reward-processing.\\
  Position CI\\
  Funding Body: National Institute of Health Research (BRC)\\
  Amount: £39, 000\\
  Awarded: December 2013
\item
  Title: Validating the Case Register Interactive Search (CRIS) system
  for large naturalistic treatment trials in youth: A two-phase study
  using Attention Deficit Hyperactivity Disorder (ADHD).\\
  Position: CI\\
  Funding Body: Biomedical Research Centre\\
  Amount: £54,483\\
  Awarded: February 2013
\item
  Title: Arterial Spin Labelling to study mood regulation in youth.\\
  Position: CI\\
  Funding Body: Biomedical Research Centre\\
  Amount: £15,000\\
  Awarded: December 2012
\item
  Title: Behavioural and Emotional Dimensions in Children, Award ID:
  089/0001\\
  Position: Co-PI (with Professor Marjorie Smith from the Institute of
  Education leading) Funding Body: Department of Health\\
  Amount: £437,194.97\\
  Awarded: 2011
\item
  Title: The developmental psychopathology of irritable mood and its
  links to depression: genetic and environmental risks,
  neuropsychological mechanisms, and hormonal influences, Project code:
  093909\\
  Position: CI Funding Body: Wellcome Trust, Intermediate Clinical
  Fellowship\\
  Amount: £439,243\\
  Awarded: April 2011
\end{itemize}

\hypertarget{mentorship-and-teaching}{%
\subsubsection{Mentorship and Teaching}\label{mentorship-and-teaching}}

\textbf{Mentorship Awards}\\
I have recently received two prizes for my mentorship at NIH, namely the
- 2019 NIH Director's Award ``for exemplary performance and significant
leadership, skill and ability in serving as a mentor.'' - 2018
Outstanding Mentor Award by the National Institutes of Mental Health

\textbf{Fulbright Fellow}

\begin{itemize}
\tightlist
\item
  Dr Neny Pervanidou, Associate Professor of Developmental Pediatrics
  spent six months in our Unit after being awarded this prestigious
  fellowship. This has laid the foundations for an ongoing
  collaboration.
\end{itemize}

\textbf{Post-doctoral fellows}

\begin{itemize}
\item
  Dr Lorena Fernandez de la Cruz, currently Assistant Professor at
  Karolinska Institutet
\item
  Dr Narun Pornpattatanangkul, currently Faculty at Ottago University
\item
  Dr Georgia O' Callaghan, currently Senior Principal Analyst at Gartner
\item
  Dr Hanna Keren, hired as faculty with Bar Ilan University at Israel
\item
  Dr Dipta Saha, currently post-doc at NIH
\item
  Dr Song Qi, currently post-doc at NIH
\item
  Dr Lucrezia Liuzzi, moved on to be Staff Scientist at NIH
\end{itemize}

\textbf{PhD Students}

\begin{itemize}
\item
  Mr Jiazhou Chen, August 2020 - Current
\item
  Ms Marie Zelenina, August 2020 - Current
\item
  Dr Nina Mikita: PhD awarded ``without corrections'' June 2016
\item
  Dr Selina Wolke: PhD awarded ``without corrections'' May 2018
\item
  Dr Pablo Vidal-Ribas: PhD awarded ``without corrections'' March 2019
  Awarded one of the 2019 Outstanding PhD Thesis Prizes from King's
  College London.
\end{itemize}

\textbf{MSc Students}

I have mentored between 2009-2015 nine MSc students for the course in
Child and Adolescent Psychiatry at King's College London. Many of them
won distinctions, such as Dr Sumudu Ferdinando who came first in her
cohort for that year with top marks for her dissertation.

\textbf{Other post-graduate Students}

I have mentored between 2016 to Present more than 10 Intramural Research
Technical Assistants (IRTAs) in the intramural programme of the NIMH.
Each one of them has ended up pursuing their career of interest with
several of them joining prestigious courses in medicine in the US
(e.g.~Ms Lisa Gorham, Washington University at St Louis), Clinical
Psychology or Psychology courses (e.g.~Mr Chris Camp, Yale University).

\hypertarget{taught-courses-and-other-supervision}{%
\subsubsection{Taught Courses and other
Supervision}\label{taught-courses-and-other-supervision}}

\begin{itemize}
\item
  Exam marker at UCL's Depression and Anxiety MSc module
\item
  Lecturer at UCL's Child and Adolescent Mental Health MSc
\item
  Dissertation Supervisor at UCL's Child and Adolescent Mental Health
  MSc
\item
  Academic Training Director for Trainees in Child Psychiatry. I had
  volunteered to be the Academic Director of the Maudsley Hospital's
  Child Psychiatry Training Scheme. In this position I was responsible
  for the content and process of trainees' academic learning. This
  included organizing speaker series, modifying training content,
  assessing competencies of trainees in relation to the expected
  curriculum and devising innovative ways to improve teaching. 2010 -
  2016
\item
  Member of the Institute of Psychiatry, Psychology and Neurosience
  (IoPPN) PhD Committee represeneting the Department of Child and
  Adolescent Psychiatry, 2012- 2016
\item
  Supervisor of SpRs/ST4s at the Maudsley Hospital Rotation Scheme,
  (2010-2016) at the Maudsley National and Specialist Mood Disorder
  Service
\item
  Taught regularly at the Maudsley Hospital's Child Psychiatry Training
  Scheme on the topics of Depression and Bipolar Disorder (2010 - 2016)
\item
  Taught regularly at the (Social Genetic and Developmental Psychiatry
  MSc) at King's College London, (2012-2016)
\end{itemize}

\hypertarget{voluntary-contributions-to-the-scientific-community}{%
\subsubsection{Voluntary contributions to the Scientific
Community}\label{voluntary-contributions-to-the-scientific-community}}

\begin{itemize}
\item
  Member of the Senior Investigator and Tenure Track Investigator NIMH
  Search Committee for the recruitment of new PIs to NIH.
\item
  Organized 2018 Suicide Workshop at NIMH
\item
  Soon after my arrival at NIMH, I organized a Workshop about suicide
  recognition and prevention with four invited speakers from within the
  NIMH and elsewhere. It was attended physically by over 260 registered
  participants and we received good feedback from attendees (many of
  whom were practitioners).
\item
  Organization of Faculty Retreats. I have been an active member of the
  Faculty Retreat Committee and has so far given talks at three out of
  the four faculty retreats that have happened since my arrival at NIMH.
\item
  Securing of competitive funds for Academic Clinical Fellow positions.
  I have secured competitive funding for these academic posts of junior
  doctors for my Department in the UK for three consecutive years, since
  I took over this responsibility as Academic Programme Director in
  2013: 2014-2015, 2015-2016 and 2016-2017.
\item
  Representative of the Department at the Equality and Diversity
  Committee of my Institution. I was part of the original ``Athena
  SWAN'' team that assessed the situation in relation to women's
  academic positions at our institution (King's College London) and the
  need to improve it. I was part of the team that devised and analyzed
  results of an employee survey on this matter and communicated it to
  individuals. I also represented my Department at such meetings. Our
  team's efforts were recognized by the Government and our Institution
  was awarded a Bronze Medal for championing the role of women in STEM.
\item
  Refereeing for Journals. I am acting as referee or ad hoc editor for a
  broad range of journals including: JAACAP, JCPP, JAMA Psychiatry, Am J
  Psychiatry, Biol Psychiatry, ECAP, eLife. I referee about 2-3 papers
  per week.
\item
  Refereeing for Funding bodies. I am acting as referee for several
  funding organization including: Wellcome Trust, MRC(UK),
  MRC(Australia), Israeli Science Foundation.
\item
  Examiner for PhDs in other Universities, such as the following: Oxford
  University PhD in Psychology, Neuroimaging and Pharmacology; Cambridge
  University PhD in Psychology and Neuroimaging; Cardiff University, PhD
  in Psychology and Epidemiology; Lisbon University, PhD in
  Computational Psychiatry; Sheffield University, PhD in Developmental
  Psychology, Örebro University, Sweden.
\end{itemize}

\hypertarget{response-to-the-coronavirus-pandemic-covid-19}{%
\subsubsection{Response to the Coronavirus Pandemic
(COVID-19)}\label{response-to-the-coronavirus-pandemic-covid-19}}

My team and I have been at the forefront of the response to the
potential mental health consequences of the pandemic in the following
ways.

\begin{itemize}
\item
  Development of Resources for the measurement and tracking of
  pandemic-related psychological problems: I am on of the three people
  who developed the CRISIS initiative (Coronavirus Health Impact Survey,
  \url{http://www.crisissurvey.org} ), the other two being Drs
  Merikangas (NIMH) and Dr Milham (Child Mind Institute). This is a tool
  for parents, children and adults to track phenomenal related to the
  pandemic. It has been translated into several languages including
  Mandarin Chinese, Greek, Japanese, Italian, French, German and
  Portuguese. It is used by more then 15 teams worldwide.
\item
  Help with translation of questionnaires in different languages: I have
  organized most translations of the CRISIS tool, including the
  back-translations and attempts at swift quality assurance. I have
  myself helped with the translations into German and Greek.
\item
  Participation in international grants and projects: I have helped more
  than 5 teams by consulting on their projects and grants
  internationally, including in the USA, Australia, New Zealand, Greece,
  and the UK. These will be listed in the grants section depending on
  the outcome of the proposals.
\item
  Collection of longitudinal data within our own team: Ours was the
  first protocol to be amended at NIMH in order to collect data related
  to COVID distress. We are collecting longitudinal data (we have
  already completed 3 waves of data collection) and are expected to
  collect several more which we will be using in conjunction with our
  longitudinal data.
\end{itemize}

\hypertarget{other-activities}{%
\subsubsection{Other activities}\label{other-activities}}

\begin{itemize}
\item
  Sports: I row competitively on Concept 2 and in 2021 I ranked
  worldwide for the 40-49 age range in the 97th percentile in 2000m
  (6:36.8), 97th percentile in 5000 m (17:49.3), and 91st percentile in
  the half marathon (1:21:40.1). In the 2023 season, I raced 5000 m
  17:56.8.
\item
  Philosophy: I am working on the relevance of the work of Panajotis
  Kondylis, particularly his Macht und Entscheidung, to emotion theory,
  as well as issues related to the conceptual history of mood inspired
  from Reinhart Kosselleck's work on Begriffsgeschichte. Some of these
  results I presented at my keynote at the European Child and Adolescent
  Psychiatry Conference this year.
\end{itemize}

\hypertarget{publications}{%
\subsubsection{Publications}\label{publications}}

\textbf{Bibliographic Overview}\\
h-index 53, n\_citations = 12320 (Google Scholar May 2022)

\textbf{Monographs}

\textbf{Stringaris} A \& Taylor E (2015). Disruptive Mood: Irritability
in Children and Adolescents. New York: Oxford University Press.

\textbf{Book Chapters}

1 Prabhakar J, Nielson DM, \textbf{Stringaris} (2022) Origins of
Anhedonia in Childhood and Adolescence,In Pizzagali D (Ed) Current
Topics in Behavioural Neuroscience. Springer

2 O'Callaghan, G. \& \textbf{Stringaris}, A (2019) Reward Processing in
Adolescent Depression. In C Harmer \& T. Baune (Eds.) Cognitive
Dimensions of Major Depressive Disorder, Oxford University Press

3 \textbf{Stringaris}, A., \& Vidal-Ribas, P. (2018). Disruptive Mood
Dysregulation Disorder. In M. Ebert, J. Leckman \& I. Petrakis (Eds.),
Current Diagnosis \& Treatment Psychiatry (Third ed.): Lange Medical
Books/McGraw-Hill.

4 Zahredine N, \textbf{Stringaris} A (2018) Bipolar Illness in Children
and Adolescents, The Maudsley Prescribing Guidelines, Ed Taylor D,
Barnes TRE, Young AH. Wiley Blackwell

5 Vidal-Ribas, P., \& \textbf{Stringaris}, A. (In press). Irritability
in Mood and Anxiety Disorders. In A. K. Roy, M. A. Brotman \& E.
Leibenluft (Eds.), Irritability in Pediatric Psychopathology: Oxford
University Press.

6 Oxley C, \textbf{Stringaris} A (2018) Comorbidity of Depression and
Anxiety with ADHD. In Oxford Textbook of Attention Deficit Hyperactivity
Disorder. Ed. Banaschewski T, Coghill D, Zuddas A. Oxford University
Press

7 Mulraney M, \textbf{Stringaris} A, Taylor A (2018). Irritability,
disruptive mood and ADHD. In Oxford Textbook of Attention Deficit
Hyperactivity Disorder. Ed. Banaschewski T, Coghill D, Zuddas A. Oxford
University Press

8 Krieger FV and \textbf{Stringaris} A (2015). Temperament and
Vulnerability to Externalizing Behavior in The Oxford Handbook of
Externalizing Spectrum Disorders, Eds. Beauchaine TP and Hinshaw S. New
York: Oxford University Press.

9 \textbf{Stringaris} A (2015). Emotion regulation and emotional
disorders: conceptual issues for clinicians and neuroscientists in
Rutter's Child and Adolescent Psychiatry, Sixth Edition, Eds. Thapar A,
Pine DS, Leckman JF, Scott S, Snowling MJ, Taylor EA. Oxford:
Wiley-Blackwell.

10 \textbf{Stringaris} AK, Asherson P (2008). Molecular Genetics in
Child Psychiatry in Advances in Biological Child Psychiatry, Eds. Rohde
LA, Banaschewski, T. Basel: Karger Publishing.

11 Giora R, \textbf{Stringaris} AK (2008). Neural Substrates of
Metaphors in The Cambridge Encyclopedia of the Language Sciences (CELS),
Ed. Hogan PC. New York: Cambridge University Press.

\textbf{Book Editorship}

1 Thapar A, Pine D, \textbf{Stringaris} A, Ford T, Cresswell C, Cortese
S, Leckman J (in preparation) Rutter's Textbook of Child Psychiatry and
Psychology, Oxford, UK: Wiley Blackwell

2 Broome MR, Harland R, Owen GS, \textbf{Stringaris} A (2012). The
Maudsley Reader in Phenomenological Psychiatry. Cambridge, UK: Cambridge
University Press.

\textbf{Publications as Preprints}

1 Jangraw DC, Keren H, Bedder RL, Rutledge RB, Perreira F, Thomas AG,
Pine DS, Zheng C, Nielson, DM , \textbf{Stringaris} A (2021)
Passage-of-Time Dysphoria: A Highly Replicable Decline in Mood During
Rest and Simple Tasks that is Moderated by Depression, Psyarxiv preprint

2.Tetereva A, Li J, Deng J, \textbf{Stringaris A}, Pat N Integrating
Task-Based Functional MRI Across Tasks Markedly Boosts Prediction and
Reliability of Brain-Cognition Relationship. BioRxiv preprint

\textbf{Journal Publications (as listed in PubMed)}

Pat N, Want Y, Stringaris (2022) A Longitudinally stable, brain-based
predictive models explain the relationships of childhood intelligence
with socio-demographic, psychological and genetic factors Human Brain
Mapping (in press)

Vulser H, Lemaitre HS, \ldots, Stringaris A, \ldots Nees F (2022)
Chronotype, Longitudinal Volumetric Brain Variations Throughout
Adolescence and Depressive Symptom Development. J Am Academy Child
Adolesc Psychiatry (in press)

Pat N,Wang Y, Bartonicek A, Candia J, Stringaris A (2022) Explainable
Machine Learning Approach to Predict and Explain the Relationship
between Task-based fMRI and Individual Differences in Cognition Cerebral
Cortex (in press)

Pan P, \ldots, Stringaris A, Ernst M (2022) Longitudinal Trajectory of
the Link Between Ventral Striatum and Depression in Adolescence AJP (in
press)

Liuzzi L, Chang KK, Zheng C, Keren H, Saha D, Nielson DM, Stringaris A.
Magnetoencephalographic Correlates of Mood and Reward Dynamics in Human
Adolescents. Cereb Cortex. 2021 Dec 18:bhab417. doi:
10.1093/cercor/bhab417. Epub ahead of print. PMID: 34921602.

Gorham LS, Sadeghi N, Eisner L, Taigman J, Haynes K, Qi K, Camp CC, Fors
P, Rodriguez D, McGuire J, Garth E, Engel C, Davis M, Towbin K,
Stringaris A, Nielson DM. Clinical utility of family history of
depression for prognosis of adolescent depression severity and duration
assessed with predictive modeling. J Child Psychol Psychiatry. 2021 Nov
30. doi: 10.1111/jcpp.13547. Epub ahead of print. PMID: 34847615.

Sadeghi N, Fors P, \ldots{} Stringaris A, Nielson DM (2022) Mood and
behaviors of depressed adolescents in a longitudinal study before and
during the COVID-19 pandemic. J Am Acad Child Adolesc Psychiatry In
press

Stringaris A. Sources of normativity in childhood depression. Eur Child
Adolesc Psychiatry. 2021 Nov;30(11):1663-1665. doi:
10.1007/s00787-021-01891-7. PMID: 34687389.

Rimfeld K, Malanchini M, Arathimos R, Gidziela A, Pain O, McMillan A,
Ogden R, Webster L, Packer AE, Shakeshaft NG, Schofield KL, Pingault JB,
Allegrini AG, Stringaris A, von Stumm S, Lewis CM, Plomin R. The
consequences of a year of the COVID-19 pandemic for the mental health of
young adult twins in England and Wales. medRxiv {[}Preprint{]}. 2021 Oct
7:2021.10.07.21264655. doi: 10.1101/2021.10.07.21264655. PMID: 34642704;
PMCID: PMC8509105.

Pat N, Riglin L, Anney R, Wang Y, Barch DM, Thapar A, Stringaris A.
Motivation and Cognitive Abilities as Mediators Between Polygenic Scores
and Psychopathology in Children. J Am Acad Child Adolesc Psychiatry.
2021 Sep 7:S0890-8567(21)01363-0. doi: 10.1016/j.jaac.2021.08.019. Epub
ahead of print. PMID: 34506929.

Murphy SE, Capitão LP, Giles SLC, Cowen PJ, Stringaris A, Harmer CJ. The
knowns and unknowns of SSRI treatment in young people with depression
and anxiety: efficacy, predictors, and mechanisms of action. Lancet
Psychiatry. 2021 Sep;8(9):824-835. doi: 10.1016/S2215-0366(21)00154-1.
PMID: 34419187.

Keren H, Zheng C, Jangraw DC, Chang K, Vitale A, Rutledge RB, Pereira F,
Nielson DM, Stringaris A. The temporal representation of experience in
subjective mood. Elife. 2021 Jun 15;10:e62051. doi: 10.7554/eLife.62051.
PMID: 34128464; PMCID: PMC8241441.

Kotoula V, Stringaris A, Mackes N, Mazibuko N, Hawkins PCT, Furey M,
Curran HV, Mehta MA. Ketamine Modulates the Neural Correlates of Reward
Processing in Unmedicated Patients in Remission From Depression. Biol
Psychiatry Cogn Neurosci Neuroimaging. 2021 Jun
11:S2451-9022(21)00163-4. doi: 10.1016/j.bpsc.2021.05.009. Epub ahead of
print. PMID: 34126264.

Jha MK, Minhajuddin A, Chin Fatt C, Shoptaw S, Kircanski K, Stringaris
A, Leibenluft E, Trivedi M. Irritability as an independent predictor of
concurrent and future suicidal ideation in adults with stimulant use
disorder: Findings from the STRIDE study. J Affect Disord. 2021 Sep
1;292:108-113. doi: 10.1016/j.jad.2021.04.019. Epub 2021 May 3. PMID:
34111690.

Oberman LM, Hynd M, Nielson DM, Towbin KE, Lisanby SH, Stringaris A.
Repetitive Transcranial Magnetic Stimulation for Adolescent Major
Depressive Disorder: A Focus on Neurodevelopment. Front Psychiatry. 2021
Apr 13;12:642847. doi: 10.3389/fpsyt.2021.642847. PMID: 33927653; PMCID:
PMC8076574.

Nikolaidis A, Paksarian D, Alexander L, Derosa J, Dunn J, Nielson DM,
Droney I, Kang M, Douka I, Bromet E, Milham M, Stringaris A, Merikangas
KR. The Coronavirus Health and Impact Survey (CRISIS) reveals
reproducible correlates of pandemic-related mood states across the
Atlantic. Sci Rep.~2021 Apr 14;11(1):8139. doi:
10.1038/s41598-021-87270-3. PMID: 33854103; PMCID: PMC8046981.

Toenders YJ, Kottaram A, Dinga R, Davey CG, Banaschewski T, Bokde ALW,
Quinlan EB, Desrivières S, Flor H, Grigis A, Garavan H, Gowland P, Heinz
A, Brühl R, Martinot JL, Paillère Martinot ML, Nees F, Orfanos DP,
Lemaitre H, Paus T, Poustka L, Hohmann S, Fröhner JH, Smolka MN, Walter
H, Whelan R, Stringaris A, van Noort B, Penttilä J, Grimmer Y, Insensee
C, Becker A, Schumann G; IMAGEN Consortium, Schmaal L. Predicting
Depression Onset in Young People Based on Clinical, Cognitive,
Environmental, and Neurobiological Data. Biol Psychiatry Cogn Neurosci
Neuroimaging. 2021 Mar 19:S2451-9022(21)00082-3. doi:
10.1016/j.bpsc.2021.03.005. Epub ahead of print. PMID: 33753312.

Vidal-Ribas P, Stringaris A. How and Why Are Irritability and Depression
Linked? Child Adolesc Psychiatr Clin N Am. 2021 Apr;30(2):401-414. doi:
10.1016/j.chc.2020.10.009. PMID: 33743947; PMCID: PMC7988746.

Hoffmann MS, Brunoni AR, Stringaris A, Viana MC, Lotufo PA, Benseñor IM,
Salum GA. Common and specific aspects of anxiety and depression and the
metabolic syndrome. J Psychiatr Res. 2021 May;137:117-125. doi:
10.1016/j.jpsychires.2021.02.052. Epub 2021 Mar 1. PMID: 33677215.

Tschorn M, Lorenz RC, O'Reilly PF, Reichenberg A, Banaschewski T, Bokde
ALW, Quinlan EB, Desrivières S, Flor H, Grigis A, Garavan H, Gowland P,
Ittermann B, Martinot JL, Artiges E, Nees F, Papadopoulos Orfanos D,
Poustka L, Millenet S, Fröhner JH, Smolka MN, Walter H, Whelan R,
Schumann G, Heinz A, Rapp MA; IMAGEN Consortium. Differential predictors
for alcohol use in adolescents as a function of familial risk. Transl
Psychiatry. 2021 Mar 4;11(1):157. doi: 10.1038/s41398-021-01260-7. PMID:
33664233; PMCID: PMC7933140.

Rimfeld K, Malanchini M, Allegrini AG, Packer AE, McMillan A, Ogden R,
Webster L, Shakeshaft NG, Schofield KL, Pingault JB, Stringaris A, von
Stumm S, Plomin R. Genetic Correlates of Psychological Responses to the
COVID-19 Crisis in Young Adult Twins in Great Britain. Behav Genet. 2021
Mar;51(2):110-124. doi: 10.1007/s10519-021-10050-2. Epub 2021 Feb 24.
PMID: 33624124; PMCID: PMC7902241.

Vidal-Ribas P, Janiri D, Doucet GE, Pornpattananangkul N, Nielson DM,
Frangou S, Stringaris A. Multimodal Neuroimaging of Suicidal Thoughts
and Behaviors in a U.S. Population-Based Sample of School-Age Children.
Am J Psychiatry. 2021 Apr 1;178(4):321-332. doi:
10.1176/appi.ajp.2020.20020120. Epub 2021 Jan 21. PMID: 33472387; PMCID:
PMC8016742.

Sugaya LS, Kircanski K, Stringaris A, Polanczyk GV, Leibenluft E.
Validation of an irritability measure in preschoolers in school-based
and clinical Brazilian samples. Eur Child Adolesc Psychiatry. 2021 Jan
2. doi: 10.1007/s00787-020-01701-6. Epub ahead of print. PMID: 33389159.

Sciberras E, Patel P, Stokes MA, Coghill D, Middeldorp CM, Bellgrove MA,
Becker SP, Efron D, Stringaris A, Faraone SV, Bellows ST, Quach J,
Banaschewski T, McGillivray J, Hutchinson D, Silk TJ, Melvin G, Wood AG,
Jackson A, Loram G, Engel L, Montgomery A, Westrupp E. Physical Health,
Media Use, and Mental Health in Children and Adolescents With ADHD
During the COVID-19 Pandemic in Australia. J Atten Disord. 2022
Feb;26(4):549-562. doi: 10.1177/1087054720978549. Epub 2020 Dec 17.
PMID: 33331195; PMCID: PMC8785303.

Robinson L, Zhang Z, Jia T, Bobou M, Roach A, Campbell I, Irish M,
Quinlan EB, Tay N, Barker ED, Banaschewski T, Bokde ALW, Grigis A,
Garavan H, Heinz A, Ittermann B, Martinot JL, Stringaris A, Penttilä J,
van Noort B, Grimmer Y, Martinot MP, Insensee C, Becker A, Nees F,
Orfanos DP, Paus T, Poustka L, Hohmann S, Fröhner JH, Smolka MN, Walter
H, Whelan R, Schumann G, Schmidt U, Desrivières S; IMAGEN Consortium.
Association of Genetic and Phenotypic Assessments With Onset of
Disordered Eating Behaviors and Comorbid Mental Health Problems Among
Adolescents. JAMA Netw Open. 2020 Dec 1;3(12):e2026874. doi:
10.1001/jamanetworkopen.2020.26874. PMID: 33263759; PMCID: PMC7711322.

Xie C, Jia T, Rolls ET, Robbins TW, Sahakian BJ, Zhang J, Liu Z, Cheng
W, Luo Q, Zac Lo CY, Wang H, Banaschewski T, Barker GJ, Bokde ALW,
Büchel C, Quinlan EB, Desrivières S, Flor H, Grigis A, Garavan H,
Gowland P, Heinz A, Hohmann S, Ittermann B, Martinot JL, Paillère
Martinot ML, Nees F, Orfanos DP, Paus T, Poustka L, Fröhner JH, Smolka
MN, Walter H, Whelan R, Schumann G, Feng J; IMAGEN Consortium. Reward
Versus Nonreward Sensitivity of the Medial Versus Lateral Orbitofrontal
Cortex Relates to the Severity of Depressive Symptoms. Biol Psychiatry
Cogn Neurosci Neuroimaging. 2021 Mar;6(3):259-269. doi:
10.1016/j.bpsc.2020.08.017. Epub 2020 Sep 10. PMID: 33221327.

Leigh E, Lee A, Brown HM, Pisano S, Stringaris A. A Prospective Study of
Rumination and Irritability in Youth. J Abnorm Child Psychol. 2020
Dec;48(12):1581-1589. doi: 10.1007/s10802-020-00706-8. Epub 2020 Oct 1.
PMID: 33001331; PMCID: PMC7554009.

Nikolaidis A, Paksarian D, Alexander L, Derosa J, Dunn J, Nielson DM,
Droney I, Kang M, Douka I, Bromet E, Milham M, Stringaris A, Merikangas
KR. The Coronavirus Health and Impact Survey (CRISIS) reveals
reproducible correlates of pandemic-related mood states across the
Atlantic. medRxiv {[}Preprint{]}. 2020 Aug 27:2020.08.24.20181123. doi:
10.1101/2020.08.24.20181123. Update in: Sci Rep. 2021 Apr 14;11(1):8139.
PMID: 32869041; PMCID: PMC7457620.

Chaarani B, Kan KJ, Mackey S, Spechler PA, Potter A, Banaschewski T,
Millenet S, Bokde ALW, Bromberg U, Büchel C, Cattrell A, Conrod PJ,
Desrivières S, Flor H, Frouin V, Gallinat J, Gowland P, Heinz A,
Ittermann B, Martinot JL, Nees F, Paus T, Poustka L, Smolka MN, Walter
H, Whelan R, Stringaris A, Higgins ST, Schumann G, Garavan H, Althoff
RR; IMAGEN Consortium. Neural Correlates of Adolescent Irritability and
Its Comorbidity With Psychiatric Disorders. J Am Acad Child Adolesc
Psychiatry. 2020 Dec;59(12):1371-1379. doi: 10.1016/j.jaac.2019.11.028.
Epub 2020 Aug 27. PMID: 32860907.

Nielson DM, Keren H, O'Callaghan G, Jackson SM, Douka I, Vidal-Ribas P,
Pornpattananangkul N, Camp CC, Gorham LS, Wei C, Kirwan S, Zheng CY,
Stringaris A. Great Expectations: A Critical Review of and Suggestions
for the Study of Reward Processing as a Cause and Predictor of
Depression. Biol Psychiatry. 2021 Jan 15;89(2):134-143. doi:
10.1016/j.biopsych.2020.06.012. Epub 2020 Jun 17. PMID: 32797941.

Zhang Z, Robinson L, Jia T, Quinlan EB, Tay N, Chu C, Barker ED,
Banaschewski T, Barker GJ, Bokde ALW, Flor H, Grigis A, Garavan H,
Gowland P, Heinz A, Ittermann B, Martinot JL, Stringaris A, Penttilä J,
van Noort B, Grimmer Y, Paillère Martinot ML, Isensee C, Becker A, Nees
F, Orfanos DP, Paus T, Poustka L, Hohmann S, Fröhner JH, Smolka MN,
Walter H, Whelan R, Schumann G, Schmidt U, Desrivières S. Development of
Disordered Eating Behaviors and Comorbid Depressive Symptoms in
Adolescence: Neural and Psychopathological Predictors. Biol Psychiatry.
2021 Dec 15;90(12):853-862. doi: 10.1016/j.biopsych.2020.06.003. Epub
2020 Jun 10. PMID: 32778392.

Rimfeld K, Malancini M, Allegrini A, Packer AE, McMillan A, Ogden R,
Webster L, Shakeshaft NG, Schofield KL, Pingault JB, Stringaris A, von
Stumm S, Plomin R. Genetic correlates of psychological responses to the
COVID-19 crisis in young adult twins in Great Britain. Res Sq
{[}Preprint{]}. 2020 May 27:rs.3.rs-31853. doi:
10.21203/rs.3.rs-31853/v1. Update in: Behav Genet. 2021 Feb 24;: PMID:
32702738; PMCID: PMC7336701.

Monzani B, Vidal-Ribas P, Turner C, Krebs G, Stokes C, Heyman I,
Mataix-Cols D, Stringaris A. The Role of Paternal Accommodation of
Paediatric OCD Symptoms: Patterns and Implications for Treatment
Outcomes. J Abnorm Child Psychol. 2020 Oct;48(10):1313-1323. doi:
10.1007/s10802-020-00678-9. PMID: 32683586; PMCID: PMC7445192.

Jha MK, Minhajuddin A, Chin Fatt C, Kircanski K, Stringaris A,
Leibenluft E, Trivedi MH. Association between irritability and suicidal
ideation in three clinical trials of adults with major depressive
disorder. Neuropsychopharmacology. 2020 Dec;45(13):2147-2154. doi:
10.1038/s41386-020-0769-x. Epub 2020 Jul 14. PMID: 32663842; PMCID:
PMC7784964.

Romani-Sponchiado A, Jordan MR, Stringaris A, Salum GA. Distinct
correlates of empathy and compassion with burnout and affective symptoms
in health professionals and students. Braz J Psychiatry. 2021
Mar-Apr;43(2):186-188. doi: 10.1590/1516-4446-2020-0941. PMID: 32638919;
PMCID: PMC8023162.

Pornpattananangkul N, Leibenluft E, Pine DS, Stringaris A. Notice of
Retraction and Replacement. Pornpattananangkul et al.~Association
between childhood anhedonia and alterations in large-scale resting-state
networks and task-evoked activation. JAMA Psychiatry.
2019;76(6):624-633. JAMA Psychiatry. 2020 Oct 1;77(10):1085-1086. doi:
10.1001/jamapsychiatry.2020.1367. PMID: 32629467.

Harrewijn A, Vidal-Ribas P, Clore-Gronenborn K, Jackson SM, Pisano S,
Pine DS, Stringaris A. Associations between brain activity and
endogenous and exogenous cortisol - A systematic review.
Psychoneuroendocrinology. 2020 Oct;120:104775. doi:
10.1016/j.psyneuen.2020.104775. Epub 2020 Jun 18. PMID: 32592873; PMCID:
PMC7502528.

Morris AC, Macdonald A, Moghraby O, Stringaris A, Hayes RD, Simonoff E,
Ford T, Downs JM. Sociodemographic factors associated with routine
outcome monitoring: a historical cohort study of 28,382 young people
accessing child and adolescent mental health services. Child Adolesc
Ment Health. 2021 Feb;26(1):56-64. doi: 10.1111/camh.12396. Epub 2020
Jun 16. PMID: 32544982.

Modabbernia A, Reichenberg A, Ing A, Moser DA, Doucet GE, Artiges E,
Banaschewski T, Barker GJ, Becker A, Bokde ALW, Quinlan EB, Desrivières
S, Flor H, Fröhner JH, Garavan H, Gowland P, Grigis A, Grimmer Y, Heinz
A, Insensee C, Ittermann B, Martinot JL, Martinot MP, Millenet S, Nees
F, Orfanos DP, Paus T, Penttilä J, Poustka L, Smolka MN, Stringaris A,
van Noort BM, Walter H, Whelan R, Schumann G, Frangou S; IMAGEN
Consortium. Linked patterns of biological and environmental covariation
with brain structure in adolescence: a population- based longitudinal
study. Mol Psychiatry. 2021 Sep;26(9):4905-4918. doi:
10.1038/s41380-020-0757-x. Epub 2020 May 22. PMID: 32444868; PMCID:
PMC7981783.

Grasby KL, Jahanshad N, Painter JN, Colodro-Conde L, Bralten J, Hibar
DP, Lind PA, Pizzagalli F, Ching CRK, McMahon MAB, Shatokhina N, Zsembik
LCP, Thomopoulos SI, Zhu AH, Strike LT, Agartz I, Alhusaini S, Almeida
MAA, Alnæs D, Amlien IK, Andersson M, Ard T, Armstrong NJ, Ashley-Koch
A, Atkins JR, Bernard M, Brouwer RM, Buimer EEL, Bülow R, Bürger C,
Cannon DM, Chakravarty M, Chen Q, Cheung JW, Couvy-Duchesne B, Dale AM,
Dalvie S, de Araujo TK, de Zubicaray GI, de Zwarte SMC, den Braber A,
Doan NT, Dohm K, Ehrlich S, Engelbrecht HR, Erk S, Fan CC, Fedko IO,
Foley SF, Ford JM, Fukunaga M, Garrett ME, Ge T, Giddaluru S, Goldman
AL, Green MJ, Groenewold NA, Grotegerd D, Gurholt TP, Gutman BA, Hansell
NK, Harris MA, Harrison MB, Haswell CC, Hauser M, Herms S, Heslenfeld
DJ, Ho NF, Hoehn D, Hoffmann P, Holleran L, Hoogman M, Hottenga JJ,
Ikeda M, Janowitz D, Jansen IE, Jia T, Jockwitz C, Kanai R, Karama S,
Kasperaviciute D, Kaufmann T, Kelly S, Kikuchi M, Klein M, Knapp M,
Knodt AR, Krämer B, Lam M, Lancaster TM, Lee PH, Lett TA, Lewis LB,
Lopes-Cendes I, Luciano M, Macciardi F, Marquand AF, Mathias SR, Melzer
TR, Milaneschi Y, Mirza-Schreiber N, Moreira JCV, Mühleisen TW,
Müller-Myhsok B, Najt P, Nakahara S, Nho K, Olde Loohuis LM, Orfanos DP,
Pearson JF, Pitcher TL, Pütz B, Quidé Y, Ragothaman A, Rashid FM, Reay
WR, Redlich R, Reinbold CS, Repple J, Richard G, Riedel BC, Risacher SL,
Rocha CS, Mota NR, Salminen L, Saremi A, Saykin AJ, Schlag F, Schmaal L,
Schofield PR, Secolin R, Shapland CY, Shen L, Shin J, Shumskaya E,
Sønderby IE, Sprooten E, Tansey KE, Teumer A, Thalamuthu A,
Tordesillas-Gutiérrez D, Turner JA, Uhlmann A, Vallerga CL, van der Meer
D, van Donkelaar MMJ, van Eijk L, van Erp TGM, van Haren NEM, van Rooij
D, van Tol MJ, Veldink JH, Verhoef E, Walton E, Wang M, Wang Y, Wardlaw
JM, Wen W, Westlye LT, Whelan CD, Witt SH, Wittfeld K, Wolf C, Wolfers
T, Wu JQ, Yasuda CL, Zaremba D, Zhang Z, Zwiers MP, Artiges E, Assareh
AA, Ayesa-Arriola R, Belger A, Brandt CL, Brown GG, Cichon S, Curran JE,
Davies GE, Degenhardt F, Dennis MF, Dietsche B, Djurovic S, Doherty CP,
Espiritu R, Garijo D, Gil Y, Gowland PA, Green RC, Häusler AN, Heindel
W, Ho BC, Hoffmann WU, Holsboer F, Homuth G, Hosten N, Jack CR Jr, Jang
M, Jansen A, Kimbrel NA, Kolskår K, Koops S, Krug A, Lim KO, Luykx JJ,
Mathalon DH, Mather KA, Mattay VS, Matthews S, Mayoral Van Son J, McEwen
SC, Melle I, Morris DW, Mueller BA, Nauck M, Nordvik JE, Nöthen MM,
O'Leary DS, Opel N, Martinot MP, Pike GB, Preda A, Quinlan EB, Rasser
PE, Ratnakar V, Reppermund S, Steen VM, Tooney PA, Torres FR, Veltman
DJ, Voyvodic JT, Whelan R, White T, Yamamori H, Adams HHH, Bis JC,
Debette S, Decarli C, Fornage M, Gudnason V, Hofer E, Ikram MA, Launer
L, Longstreth WT, Lopez OL, Mazoyer B, Mosley TH, Roshchupkin GV,
Satizabal CL, Schmidt R, Seshadri S, Yang Q; Alzheimer's Disease
Neuroimaging Initiative; CHARGE Consortium; EPIGEN Consortium; IMAGEN
Consortium; SYS Consortium; Parkinson's Progression Markers Initiative,
Alvim MKM, Ames D, Anderson TJ, Andreassen OA, Arias-Vasquez A, Bastin
ME, Baune BT, Beckham JC, Blangero J, Boomsma DI, Brodaty H, Brunner HG,
Buckner RL, Buitelaar JK, Bustillo JR, Cahn W, Cairns MJ, Calhoun V,
Carr VJ, Caseras X, Caspers S, Cavalleri GL, Cendes F, Corvin A,
Crespo-Facorro B, Dalrymple-Alford JC, Dannlowski U, de Geus EJC, Deary
IJ, Delanty N, Depondt C, Desrivières S, Donohoe G, Espeseth T,
Fernández G, Fisher SE, Flor H, Forstner AJ, Francks C, Franke B, Glahn
DC, Gollub RL, Grabe HJ, Gruber O, Håberg AK, Hariri AR, Hartman CA,
Hashimoto R, Heinz A, Henskens FA, Hillegers MHJ, Hoekstra PJ, Holmes
AJ, Hong LE, Hopkins WD, Hulshoff Pol HE, Jernigan TL, Jönsson EG, Kahn
RS, Kennedy MA, Kircher TTJ, Kochunov P, Kwok JBJ, Le Hellard S,
Loughland CM, Martin NG, Martinot JL, McDonald C, McMahon KL,
Meyer-Lindenberg A, Michie PT, Morey RA, Mowry B, Nyberg L, Oosterlaan
J, Ophoff RA, Pantelis C, Paus T, Pausova Z, Penninx BWJH, Polderman
TJC, Posthuma D, Rietschel M, Roffman JL, Rowland LM, Sachdev PS, Sämann
PG, Schall U, Schumann G, Scott RJ, Sim K, Sisodiya SM, Smoller JW,
Sommer IE, St Pourcain B, Stein DJ, Toga AW, Trollor JN, Van der Wee
NJA, van 't Ent D, Völzke H, Walter H, Weber B, Weinberger DR, Wright
MJ, Zhou J, Stein JL, Thompson PM, Medland SE; Enhancing NeuroImaging
Genetics through Meta-Analysis Consortium (ENIGMA)---Genetics working
group. The genetic architecture of the human cerebral cortex. Science.
2020 Mar 20;367(6484):eaay6690. doi: 10.1126/science.aay6690. Erratum
in: Science. 2021 Oct 22;374(6566):eabm7211. PMID: 32193296; PMCID:
PMC7295264.

Lewis KM, Matsumoto C, Cardinale E, Jones EL, Gold AL, Stringaris A,
Leibenluft E, Pine DS, Brotman MA. Self-Efficacy As a Target for
Neuroscience Research on Moderators of Treatment Outcomes in Pediatric
Anxiety. J Child Adolesc Psychopharmacol. 2020 May;30(4):205-214. doi:
10.1089/cap.2019.0130. Epub 2020 Mar 11. PMID: 32167803; PMCID:
PMC7360109.

Haller SP, Kircanski K, Stringaris A, Clayton M, Bui H, Agorsor C,
Cardenas SI, Towbin KE, Pine DS, Leibenluft E, Brotman MA. The Clinician
Affective Reactivity Index: Validity and Reliability of a
Clinician-Rated Assessment of Irritability. Behav Ther. 2020
Mar;51(2):283-293. doi: 10.1016/j.beth.2019.10.005. Epub 2019 Nov 27.
PMID: 32138938; PMCID: PMC7060970.

Cuijpers P, Stringaris A, Wolpert M. Treatment outcomes for depression:
challenges and opportunities. Lancet Psychiatry. 2020 Nov;7(11):925-927.
doi: 10.1016/S2215-0366(20)30036-5. Epub 2020 Feb 17. PMID: 32078823.

Dwyer JB, Stringaris A, Brent DA, Bloch MH. Annual Research Review:
Defining and treating pediatric treatment-resistant depression. J Child
Psychol Psychiatry. 2020 Mar;61(3):312-332. doi: 10.1111/jcpp.13202.
Epub 2020 Feb 4. PMID: 32020643; PMCID: PMC8314167.

Villalta L, Khadr S, Chua KC, Kramer T, Clarke V, Viner RM, Stringaris
A, Smith P. Complex post-traumatic stress symptoms in female
adolescents: the role of emotion dysregulation in impairment and trauma
exposure after an acute sexual assault. Eur J Psychotraumatol. 2020 Jan
10;11(1):1710400. doi: 10.1080/20008198.2019.1710400. PMID: 32002143;
PMCID: PMC6968575.

Frere PB, Vetter NC, Artiges E, Filippi I, Miranda R, Vulser H,
Paillère- Martinot ML, Ziesch V, Conrod P, Cattrell A, Walter H,
Gallinat J, Bromberg U, Jurk S, Menningen E, Frouin V, Papadopoulos
Orfanos D, Stringaris A, Penttilä J, van Noort B, Grimmer Y, Schumann G,
Smolka MN, Martinot JL, Lemaître H; Imagen consortium. Sex effects on
structural maturation of the limbic system and outcomes on emotional
regulation during adolescence. Neuroimage. 2020 Apr 15;210:116441. doi:
10.1016/j.neuroimage.2019.116441. Epub 2019 Dec 4. PMID: 31811901.

Stringaris A. Editorial: Are computers going to take over: implications
of machine learning and computational psychiatry for trainees and
practising clinicians. J Child Psychol Psychiatry. 2019
Dec;60(12):1251-1253. doi: 10.1111/jcpp.13168. PMID: 31724195.

Ing A, Sämann PG, Chu C, Tay N, Biondo F, Robert G, Jia T, Wolfers T,
Desrivières S, Banaschewski T, Bokde ALW, Bromberg U, Büchel C, Conrod
P, Fadai T, Flor H, Frouin V, Garavan H, Spechler PA, Gowland P, Grimmer
Y, Heinz A, Ittermann B, Kappel V, Martinot JL, Meyer-Lindenberg A,
Millenet S, Nees F, van Noort B, Orfanos DP, Martinot MP, Penttilä J,
Poustka L, Quinlan EB, Smolka MN, Stringaris A, Struve M, Veer IM,
Walter H, Whelan R, Andreassen OA, Agartz I, Lemaitre H, Barker ED,
Ashburner J, Binder E, Buitelaar J, Marquand A, Robbins TW, Schumann G;
IMAGEN Consortium. Identification of neurobehavioural symptom groups
based on shared brain mechanisms. Nat Hum Behav. 2019
Dec;3(12):1306-1318. doi: 10.1038/s41562-019-0738-8. Epub 2019 Oct 7.
PMID: 31591521.

Reedtz C, van Doesum K, Signorini G, Lauritzen C, van Amelsvoort T, van
Santvoort F, Young AH, Conus P, Musil R, Schulze T, Berk M, Stringaris
A, Piché G, de Girolamo G. Promotion of Wellbeing for Children of
Parents With Mental Illness: A Model Protocol for Research and
Intervention. Front Psychiatry. 2019 Sep 6;10:606. doi:
10.3389/fpsyt.2019.00606. PMID: 31572227; PMCID: PMC6752481.

Galinowski A, Miranda R, Lemaitre H, Artiges E, Paillère Martinot ML,
Filippi I, Penttilä J, Grimmer Y, van Noort BM, Stringaris A, Becker A,
Isensee C, Struve M, Fadai T, Kappel V, Goodman R, Banaschewski T, Bokde
ALW, Bromberg U, Brühl R, Büchel C, Cattrell A, Conrod P, Desrivières S,
Flor H, Fröhner JH, Frouin V, Gallinat J, Garavan H, Gowland P, Heinz A,
Hohmann S, Jurk S, Millenet S, Nees F, Papadopoulos-Orfanos D, Poustka
L, Quinlan EB, Smolka MN, Walter H, Whelan R, Schumann G, Martinot JL;
IMAGEN Consortium. Heavy drinking in adolescents is associated with
change in brainstem microstructure and reward sensitivity. Addict Biol.
2020 May;25(3):e12781. doi: 10.1111/adb.12781. Epub 2019 Jul 21. PMID:
31328396.

Vidal-Ribas P, Benson B, Vitale AD, Keren H, Harrewijn A, Fox NA, Pine
DS, Stringaris A. Bidirectional Associations Between Stress and Reward
Processing in Children and Adolescents: A Longitudinal Neuroimaging
Study. Biol Psychiatry Cogn Neurosci Neuroimaging. 2019
Oct;4(10):893-901. doi: 10.1016/j.bpsc.2019.05.012. Epub 2019 Jun 3.
PMID: 31324591; PMCID: PMC6783352.

Riglin L, Eyre O, Thapar AK, Stringaris A, Leibenluft E, Pine DS,
Tilling K, Davey Smith G, O'Donovan MC, Thapar A. Identifying Novel
Types of Irritability Using a Developmental Genetic Approach. Am J
Psychiatry. 2019 Aug 1;176(8):635-642. doi:
10.1176/appi.ajp.2019.18101134. Epub 2019 Jul 1. PMID: 31256611; PMCID:
PMC6677571.

Towbin K, Vidal-Ribas P, Brotman MA, Pickles A, Miller KV, Kaiser A,
Vitale AD, Engel C, Overman GP, Davis M, Lee B, McNeil C, Wheeler W,
Yokum CH, Haring CT, Roule A, Wambach CG, Sharif-Askary B, Pine DS,
Leibenluft E, Stringaris A. A Double-Blind Randomized Placebo-Controlled
Trial of Citalopram Adjunctive to Stimulant Medication in Youth With
Chronic Severe Irritability. J Am Acad Child Adolesc Psychiatry. 2020
Mar;59(3):350-361. doi: 10.1016/j.jaac.2019.05.015. Epub 2019 May 23.
PMID: 31128268.

Chaarani B, Kan KJ, Mackey S, Spechler PA, Potter A, Orr C, D'Alberto N,
Hudson KE, Banaschewski T, Bokde ALW, Bromberg U, Büchel C, Cattrell A,
Conrod PJ, Desrivières S, Flor H, Frouin V, Gallinat J, Gowland P, Heinz
A, Ittermann B, Martinot JL, Nees F, Papadopoulos-Orfanos D, Paus T,
Poustka L, Smolka MN, Walter H, Whelan R, Higgins ST, Schumann G,
Althoff RR, Stein EA, Garavan H; IMAGEN Consortium. Low Smoking
Exposure, the Adolescent Brain, and the Modulating Role of CHRNA5
Polymorphisms. Biol Psychiatry Cogn Neurosci Neuroimaging. 2019
Jul;4(7):672-679. doi: 10.1016/j.bpsc.2019.02.006. Epub 2019 Mar 15.
PMID: 31072760; PMCID: PMC6709448.

O'Callaghan G, Stringaris A. Reward Processing in Adolescent Depression
Across Neuroimaging Modalities. Z Kinder Jugendpsychiatr Psychother.
2019 Nov;47(6):535-541. doi: 10.1024/1422-4917/a000663. Epub 2019 Apr 8.
PMID: 30957688; PMCID: PMC6996129.

Eyre O, Hughes RA, Thapar AK, Leibenluft E, Stringaris A, Davey Smith G,
Stergiakouli E, Collishaw S, Thapar A. Childhood neurodevelopmental
difficulties and risk of adolescent depression: the role of
irritability. J Child Psychol Psychiatry. 2019 Aug;60(8):866-874. doi:
10.1111/jcpp.13053. Epub 2019 Mar 25. PMID: 30908655; PMCID: PMC6767365.

Pornpattananangkul N, Leibenluft E, Pine DS, Stringaris A. Association
Between Childhood Anhedonia and Alterations in Large-scale Resting-State
Networks and Task-Evoked Activation. JAMA Psychiatry. 2019 Jun
1;76(6):624-633. doi: 10.1001/jamapsychiatry.2019.0020. Retraction in:
JAMA Psychiatry. 2020 Oct 1;77(10):1085-1086. Erratum in: JAMA
Psychiatry. 2019 Apr 3;: PMID: 30865236; PMCID: PMC6552295.

Eyre O, Riglin L, Leibenluft E, Stringaris A, Collishaw S, Thapar A.
Irritability in ADHD: association with later depression symptoms. Eur
Child Adolesc Psychiatry. 2019 Oct;28(10):1375-1384. doi:
10.1007/s00787-019-01303-x. Epub 2019 Mar 5. PMID: 30834985; PMCID:
PMC6785584.

Ernst M, Benson B, Artiges E, Gorka AX, Lemaitre H, Lago T, Miranda R,
Banaschewski T, Bokde ALW, Bromberg U, Brühl R, Büchel C, Cattrell A,
Conrod P, Desrivières S, Fadai T, Flor H, Grigis A, Gallinat J, Garavan
H, Gowland P, Grimmer Y, Heinz A, Kappel V, Nees F, Papadopoulos-Orfanos
D, Penttilä J, Poustka L, Smolka MN, Stringaris A, Struve M, van Noort
BM, Walter H, Whelan R, Schumann G, Grillon C, Martinot MP, Martinot JL;
IMAGEN Consortium. Pubertal maturation and sex effects on the
default-mode network connectivity implicated in mood dysregulation.
Transl Psychiatry. 2019 Feb 25;9(1):103. doi: 10.1038/s41398-019-0433-6.
PMID: 30804326; PMCID: PMC6389927.

Stringaris A. Debate: Pediatric bipolar disorder - divided by a common
language? Child Adolesc Ment Health. 2019 Feb;24(1):106-107. doi:
10.1111/camh.12314. PMID: 32677239.

Wolke SA, Mehta MA, O'Daly O, Zelaya F, Zahreddine N, Keren H,
O'Callaghan G, Young AH, Leibenluft E, Pine DS, Stringaris A. Modulation
of anterior cingulate cortex reward and penalty signalling in
medication-naive young-adult subjects with depressive symptoms following
acute dose lurasidone. Psychol Med. 2019 Jun;49(8):1365-1377. doi:
10.1017/S0033291718003306. Epub 2019 Jan 4. PMID: 30606271; PMCID:
PMC6518385.

Stringaris A, Vidal-Ribas P. Probing the Irritability-Suicidality Nexus.
J Am Acad Child Adolesc Psychiatry. 2019 Jan;58(1):18-19. doi:
10.1016/j.jaac.2018.08.014. PMID: 30577933.

Stringaris A, Stringaris K. Editorial: Should child psychiatry be more
like paediatric oncology? J Child Psychol Psychiatry. 2018
Dec;59(12):1225-1227. doi: 10.1111/jcpp.13006. Erratum in: J Child
Psychol Psychiatry. 2019 Jan;60(1):e1. PMID: 30450645.

Tseng WL, Deveney CM, Stoddard J, Kircanski K, Frackman AE, Yi JY, Hsu
D, Moroney E, Machlin L, Donahue L, Roule A, Perhamus G, Reynolds RC,
Roberson-Nay R, Hettema JM, Towbin KE, Stringaris A, Pine DS, Brotman
MA, Leibenluft E. Brain Mechanisms of Attention Orienting Following
Frustration: Associations With Irritability and Age in Youths. Am J
Psychiatry. 2019 Jan 1;176(1):67-76. doi:
10.1176/appi.ajp.2018.18040491. Epub 2018 Oct 19. PMID: 30336704; PMCID:
PMC6408218.

Buckley V, Krebs G, Bowyer L, Jassi A, Goodman R, Clark B, Stringaris A.
Innovations in Practice: Body dysmorphic disorder in youth - using the
Development and Well-Being Assessment as a tool to improve detection in
routine clinical practice. Child Adolesc Ment Health. 2018
Sep;23(3):291-294. doi: 10.1111/camh.12268. Epub 2018 Feb 28. PMID:
32677303.

Bolhuis K, Muetzel RL, Stringaris A, Hudziak JJ, Jaddoe VWV, Hillegers
MHJ, White T, Kushner SA, Tiemeier H. Structural Brain Connectivity in
Childhood Disruptive Behavior Problems: A Multidimensional Approach.
Biol Psychiatry. 2019 Feb 15;85(4):336-344. doi:
10.1016/j.biopsych.2018.07.005. Epub 2018 Aug 16. PMID: 30119874.

Vulser H, Paillère Martinot ML, Artiges E, Miranda R, Penttilä J,
Grimmer Y, van Noort BM, Stringaris A, Struve M, Fadai T, Kappel V,
Goodman R, Tzavara E, Massaad C, Banaschewski T, Barker GJ, Bokde ALW,
Bromberg U, Brühl R, Büchel C, Cattrell A, Conrod P, Desrivières S, Flor
H, Frouin V, Gallinat J, Garavan H, Gowland P, Heinz A, Nees F,
Papadopoulos-Orfanos D, Paus T, Poustka L, Rodehacke S, Smolka MN,
Walter H, Whelan R, Schumann G, Martinot JL, Lemaitre H; IMAGEN
Consortium. Early Variations in White Matter Microstructure and
Depression Outcome in Adolescents With Subthreshold Depression. Am J
Psychiatry. 2018 Dec 1;175(12):1255-1264. doi:
10.1176/appi.ajp.2018.17070825. Epub 2018 Aug 16. PMID: 30111185.

Bayard F, Nymberg Thunell C, Abé C, Almeida R, Banaschewski T, Barker G,
Bokde ALW, Bromberg U, Büchel C, Quinlan EB, Desrivières S, Flor H,
Frouin V, Garavan H, Gowland P, Heinz A, Ittermann B, Martinot JL,
Martinot MP, Nees F, Orfanos DP, Paus T, Poustka L, Conrod P, Stringaris
A, Struve M, Penttilä J, Kappel V, Grimmer Y, Fadai T, van Noort B,
Smolka MN, Vetter NC, Walter H, Whelan R, Schumann G, Petrovic P; IMAGEN
Consortium. Distinct brain structure and behavior related to ADHD and
conduct disorder traits. Mol Psychiatry. 2020 Nov;25(11):3020-3033. doi:
10.1038/s41380-018-0202-6. Epub 2018 Aug 14. PMID: 30108313; PMCID:
PMC7577834.

Vidal-Ribas P, Brotman MA, Salum GA, Kaiser A, Meffert L, Pine DS,
Leibenluft E, Stringaris A. Deficits in emotion recognition are
associated with depressive symptoms in youth with disruptive mood
dysregulation disorder. Depress Anxiety. 2018 Dec;35(12):1207-1217. doi:
10.1002/da.22810. Epub 2018 Jul 13. PMID: 30004611.

Wesselhoeft R, Stringaris A, Sibbersen C, Kristensen RV, Bojesen AB,
Talati A. Dimensions and subtypes of oppositionality and their relation
to comorbidity and psychosocial characteristics. Eur Child Adolesc
Psychiatry. 2019 Mar;28(3):351-365. doi: 10.1007/s00787-018-1199-8. Epub
2018 Jul 12. PMID: 30003396.

Keren H, O'Callaghan G, Vidal-Ribas P, Buzzell GA, Brotman MA,
Leibenluft E, Pan PM, Meffert L, Kaiser A, Wolke S, Pine DS, Stringaris
A. Reward Processing in Depression: A Conceptual and Meta-Analytic
Review Across fMRI and EEG Studies. Am J Psychiatry. 2018 Nov
1;175(11):1111-1120. doi: 10.1176/appi.ajp.2018.17101124. Epub 2018 Jun
20. PMID: 29921146; PMCID: PMC6345602.

Spechler PA, Allgaier N, Chaarani B, Whelan R, Watts R, Orr C, Albaugh
MD, D'Alberto N, Higgins ST, Hudson KE, Mackey S, Potter A, Banaschewski
T, Bokde ALW, Bromberg U, Büchel C, Cattrell A, Conrod PJ, Desrivières
S, Flor H, Frouin V, Gallinat J, Gowland P, Heinz A, Ittermann B,
Martinot JL, Paillère Martinot ML, Nees F, Papadopoulos Orfanos D, Paus
T, Poustka L, Smolka MN, Walter H, Schumann G, Althoff RR, Garavan H;
IMAGEN Consortium. The initiation of cannabis use in adolescence is
predicted by sex-specific psychosocial and neurobiological features. Eur
J Neurosci. 2019 Aug;50(3):2346-2356. doi: 10.1111/ejn.13989. Epub 2018
Oct 15. PMID: 29889330; PMCID: PMC7444673.

Asarnow J, Bloch MH, Brandeis D, Alexandra Burt S, Fearon P, Fombonne E,
Green J, Gregory A, Gunnar M, Halperin JM, Hollis C, Jaffee S, Klump K,
Landau S, Lesch KP, Oldehinkel AJT, Peterson B, Ramchandani P,
Sonuga-Barke E, Stringaris A, Zeanah CH. Special Editorial: Open science
and the Journal of Child Psychology \& Psychiatry - next steps? J Child
Psychol Psychiatry. 2018 Jul;59(7):826-827. doi: 10.1111/jcpp.12929.
Epub 2018 May 28. PMID: 29806217.

Keren H, Chen G, Benson B, Ernst M, Leibenluft E, Fox NA, Pine DS,
Stringaris A. Is the encoding of Reward Prediction Error reliable during
development? Neuroimage. 2018 Sep;178:266-276. doi:
10.1016/j.neuroimage.2018.05.039. Epub 2018 May 16. PMID: 29777827;
PMCID: PMC7518449.

Humphreys KL, Schouboe SNF, Kircanski K, Leibenluft E, Stringaris A,
Gotlib IH. Irritability, Externalizing, and Internalizing
Psychopathology in Adolescence: Cross-Sectional and Longitudinal
Associations and Moderation by Sex. J Clin Child Adolesc Psychol. 2019
Sep-Oct;48(5):781-789. doi: 10.1080/15374416.2018.1460847. Epub 2018 Apr
18. PMID: 29667523; PMCID: PMC6215733.

Kircanski K, White LK, Tseng WL, Wiggins JL, Frank HR, Sequeira S, Zhang
S, Abend R, Towbin KE, Stringaris A, Pine DS, Leibenluft E, Brotman MA.
A Latent Variable Approach to Differentiating Neural Mechanisms of
Irritability and Anxiety in Youth. JAMA Psychiatry. 2018 Jun
1;75(6):631-639. doi: 10.1001/jamapsychiatry.2018.0468. PMID: 29625429;
PMCID: PMC6137523.

Brislin SJ, Patrick CJ, Flor H, Nees F, Heinrich A, Drislane LE, Yancey
JR, Banaschewski T, Bokde ALW, Bromberg U, Büchel C, Quinlan EB,
Desrivières S, Frouin V, Garavan H, Gowland P, Heinz A, Ittermann B,
Martinot JL, Martinot MP, Papadopoulos Orfanos D, Poustka L, Fröhner JH,
Smolka MN, Walter H, Whelan R, Conrod P, Stringaris A, Struve M, van
Noort B, Grimmer Y, Fadai T, Schumann G, Foell J. Extending the
Construct Network of Trait Disinhibition to the Neuroimaging Domain:
Validation of a Bridging Scale for Use in the European IMAGEN Project.
Assessment. 2019 Jun;26(4):567-581. doi: 10.1177/1073191118759748. Epub
2018 Mar 20. PMID: 29557190.

Villalta L, Smith P, Hickin N, Stringaris A. Emotion regulation
difficulties in traumatized youth: a meta-analysis and conceptual
review. Eur Child Adolesc Psychiatry. 2018 Apr;27(4):527-544. doi:
10.1007/s00787-018-1105-4. Epub 2018 Jan 27. PMID: 29380069.

Fernández de la Cruz L, Vidal-Ribas P, Zahreddine N, Mathiassen B,
Brøndbo PH, Simonoff E, Goodman R, Stringaris A. Should Clinicians Split
or Lump Psychiatric Symptoms? The Structure of Psychopathology in Two
Large Pediatric Clinical Samples from England and Norway. Child
Psychiatry Hum Dev. 2018 Aug;49(4):607-620. doi:
10.1007/s10578-017-0777-1. PMID: 29243079; PMCID: PMC6019426.

Stringaris A. Editorial: What is depression? J Child Psychol Psychiatry.
2017 Dec;58(12):1287-1289. doi: 10.1111/jcpp.12844. PMID: 29148049.

Daley D, Van Der Oord S, Ferrin M, Cortese S, Danckaerts M, Doepfner M,
Van den Hoofdakker BJ, Coghill D, Thompson M, Asherson P, Banaschewski
T, Brandeis D, Buitelaar J, Dittmann RW, Hollis C, Holtmann M, Konofal
E, Lecendreux M, Rothenberger A, Santosh P, Simonoff E, Soutullo C,
Steinhausen HC, Stringaris A, Taylor E, Wong ICK, Zuddas A, Sonuga-Barke
EJ. Practitioner Review: Current best practice in the use of parent
training and other behavioural interventions in the treatment of
children and adolescents with attention deficit hyperactivity disorder.
J Child Psychol Psychiatry. 2018 Sep;59(9):932-947. doi:
10.1111/jcpp.12825. Epub 2017 Oct 30. PMID: 29083042.

Stringaris A, Vidal-Ribas P, Brotman MA, Leibenluft E. Practitioner
Review: Definition, recognition, and treatment challenges of
irritability in young people. J Child Psychol Psychiatry. 2018
Jul;59(7):721-739. doi: 10.1111/jcpp.12823. Epub 2017 Oct 30. PMID:
29083031.

Riglin L, Eyre O, Cooper M, Collishaw S, Martin J, Langley K, Leibenluft
E, Stringaris A, Thapar AK, Maughan B, O'Donovan MC, Thapar A.
Investigating the genetic underpinnings of early-life irritability.
Transl Psychiatry. 2017 Sep 26;7(9):e1241. doi: 10.1038/tp.2017.212.
PMID: 28949337; PMCID: PMC5639253.

Pan PM, Sato JR, Salum GA, Rohde LA, Gadelha A, Zugman A, Mari J,
Jackowski A, Picon F, Miguel EC, Pine DS, Leibenluft E, Bressan RA,
Stringaris A. Ventral Striatum Functional Connectivity as a Predictor of
Adolescent Depressive Disorder in a Longitudinal Community-Based Sample.
Am J Psychiatry. 2017 Nov 1;174(11):1112-1119. doi:
10.1176/appi.ajp.2017.17040430. Epub 2017 Sep 26. PMID: 28946760.

Barker ED, Walton E, Cecil CAM, Rowe R, Jaffee SR, Maughan B, O'Connor
TG, Stringaris A, Meehan AJ, McArdle W, Relton CL, Gaunt TR. A
Methylome-Wide Association Study of Trajectories of Oppositional Defiant
Behaviors and Biological Overlap With Attention Deficit Hyperactivity
Disorder. Child Dev. 2018 Sep;89(5):1839-1855. doi: 10.1111/cdev.12957.
Epub 2017 Sep 20. PMID: 28929496; PMCID: PMC6207925.

Eyre O, Langley K, Stringaris A, Leibenluft E, Collishaw S, Thapar A.
Irritability in ADHD: Associations with depression liability. J Affect
Disord. 2017 Jun;215:281-287. doi: 10.1016/j.jad.2017.03.050. Epub 2017
Mar 25. PMID: 28363151; PMCID: PMC5409953.

Urrila AS, Artiges E, Massicotte J, Miranda R, Vulser H, Bézivin-Frere
P, Lapidaire W, Lemaître H, Penttilä J, Conrod PJ, Garavan H, Paillère
Martinot ML, Martinot JL; IMAGEN consortium. Sleep habits, academic
performance, and the adolescent brain structure. Sci Rep.~2017 Feb
9;7:41678. doi: 10.1038/srep41678. PMID: 28181512; PMCID: PMC5299428.

Brotman MA, Kircanski K, Stringaris A, Pine DS, Leibenluft E.
Irritability in Youths: A Translational Model. Am J Psychiatry. 2017 Jun
1;174(6):520-532. doi: 10.1176/appi.ajp.2016.16070839. Epub 2017 Jan 20.
PMID: 28103715.

Cortese S, Adamo N, Mohr-Jensen C, Hayes AJ, Bhatti S, Carucci S, Del
Giovane C, Atkinson LZ, Banaschewski T, Simonoff E, Zuddas A, Barbui C,
Purgato M, Steinhausen HC, Shokraneh F, Xia J, Cipriani A, Coghill D;
European ADHD Guidelines Group (EAGG). Comparative efficacy and
tolerability of pharmacological interventions for
attention-deficit/hyperactivity disorder in children, adolescents and
adults: protocol for a systematic review and network meta-analysis. BMJ
Open. 2017 Jan 10;7(1):e013967. doi: 10.1136/bmjopen-2016-013967. PMID:
28073796; PMCID: PMC5253538.

Cortese S, Brandeis D, Holtmann M, Sonuga-Barke EJ; European ADHD
Guidelines Group (EAGG). The European ADHD Guidelines Group replies. J
Am Acad Child Adolesc Psychiatry. 2016 Dec;55(12):1092-1093. doi:
10.1016/j.jaac.2016.09.492. PMID: 27871646.

Stringaris A. Editorial: Boredom and developmental psychopathology. J
Child Psychol Psychiatry. 2016 Dec;57(12):1335-1336. doi:
10.1111/jcpp.12664. PMID: 27859345.

Salum GA, Mogg K, Bradley BP, Stringaris A, Gadelha A, Pan PM, Rohde LA,
Polanczyk GV, Manfro GG, Pine DS, Leibenluft E. Association between
irritability and bias in attention orienting to threat in children and
adolescents. J Child Psychol Psychiatry. 2017 May;58(5):595-602. doi:
10.1111/jcpp.12659. Epub 2016 Oct 26. PMID: 27782299.

Kircanski K, Zhang S, Stringaris A, Wiggins JL, Towbin KE, Pine DS,
Leibenluft E, Brotman MA. Empirically derived patterns of psychiatric
symptoms in youth: A latent profile analysis. J Affect Disord. 2017
Jul;216:109-116. doi: 10.1016/j.jad.2016.09.016. Epub 2016 Sep 21. PMID:
27692699; PMCID: PMC5360533.

Koukounari A, Stringaris A, Maughan B. Pathways from maternal depression
to young adult offspring depression: an exploratory longitudinal
mediation analysis. Int J Methods Psychiatr Res. 2017 Jun;26(2):e1520.
doi: 10.1002/mpr.1520. Epub 2016 Jul 29. PMID: 27469020; PMCID:
PMC5484332.

Mikita N, Simonoff E, Pine DS, Goodman R, Artiges E, Banaschewski T,
Bokde AL, Bromberg U, Büchel C, Cattrell A, Conrod PJ, Desrivières S,
Flor H, Frouin V, Gallinat J, Garavan H, Heinz A, Ittermann B, Jurk S,
Martinot JL, Paillère Martinot ML, Nees F, Papadopoulos Orfanos D, Paus
T, Poustka L, Smolka MN, Walter H, Whelan R, Schumann G, Stringaris A.
Disentangling the autism-anxiety overlap: fMRI of reward processing in a
community-based longitudinal study. Transl Psychiatry. 2016 Jun
28;6(6):e845. doi: 10.1038/tp.2016.107. PMID: 27351599; PMCID:
PMC4931605.

Vidal-Ribas P, Brotman MA, Valdivieso I, Leibenluft E, Stringaris A. The
Status of Irritability in Psychiatry: A Conceptual and Quantitative
Review. J Am Acad Child Adolesc Psychiatry. 2016 Jul;55(7):556-70. doi:
10.1016/j.jaac.2016.04.014. Epub 2016 May 6. PMID: 27343883; PMCID:
PMC4927461.

Medford N, Sierra M, Stringaris A, Giampietro V, Brammer MJ, David AS.
Emotional Experience and Awareness of Self: Functional MRI Studies of
Depersonalization Disorder. Front Psychol. 2016 Jun 2;7:432. doi:
10.3389/fpsyg.2016.00432. PMID: 27313548; PMCID: PMC4890597.

Cortese S, Ferrin M, Brandeis D, Holtmann M, Aggensteiner P, Daley D,
Santosh P, Simonoff E, Stevenson J, Stringaris A, Sonuga-Barke EJ;
European ADHD Guidelines Group (EAGG). Neurofeedback for
Attention-Deficit/Hyperactivity Disorder: Meta-Analysis of Clinical and
Neuropsychological Outcomes From Randomized Controlled Trials. J Am Acad
Child Adolesc Psychiatry. 2016 Jun;55(6):444-55. doi:
10.1016/j.jaac.2016.03.007. Epub 2016 Apr 1. PMID: 27238063.

Kaurin A, Egloff B, Stringaris A, Wessa M. Only complementary voices
tell the truth: a reevaluation of validity in multi-informant approaches
of child and adolescent clinical assessments. J Neural Transm (Vienna).
2016 Aug;123(8):981-90. doi: 10.1007/s00702-016-1543-4. Epub 2016 Apr
27. PMID: 27118025.

Stringaris A. Editorial: Neuroimaging in clinical psychiatry--when will
the pay off begin? J Child Psychol Psychiatry. 2015 Dec;56(12):1263-5.
doi: 10.1111/jcpp.12490. PMID: 26768523.

Algorta GP, Dodd AL, Stringaris A, Youngstrom EA. Diagnostic efficiency
of the SDQ for parents to identify ADHD in the UK: a ROC analysis. Eur
Child Adolesc Psychiatry. 2016 Sep;25(9):949-57. doi:
10.1007/s00787-015-0815-0. Epub 2016 Jan 14. PMID: 26762184; PMCID:
PMC4990620.

Shaw P, Stringaris A, Nigg J, Leibenluft E. Emotion Dysregulation in
Attention Deficit Hyperactivity Disorder. Focus (Am Psychiatr Publ).
2016 Jan;14(1):127-144. doi: 10.1176/appi.focus.140102. Epub 2015 Dec
24. PMID: 31997948; PMCID: PMC6524448.

Sonuga-Barke EJ, Cortese S, Fairchild G, Stringaris A. Annual Research
Review: Transdiagnostic neuroscience of child and adolescent mental
disorders-- differentiating decision making in
attention-deficit/hyperactivity disorder, conduct disorder, depression,
and anxiety. J Child Psychol Psychiatry. 2016 Mar;57(3):321-49. doi:
10.1111/jcpp.12496. Epub 2015 Dec 26. PMID: 26705858; PMCID: PMC4762324.

Hoffmann MS, Leibenluft E, Stringaris A, Laporte PP, Pan PM, Gadelha A,
Manfro GG, Miguel EC, Rohde LA, Salum GA. Positive Attributes Buffer the
Negative Associations Between Low Intelligence and High Psychopathology
With Educational Outcomes. J Am Acad Child Adolesc Psychiatry. 2016
Jan;55(1):47-53. doi: 10.1016/j.jaac.2015.10.013. Epub 2015 Nov 10.
PMID: 26703909; PMCID: PMC4695393.

Benarous X, Mikita N, Goodman R, Stringaris A. Distinct relationships
between social aptitude and dimensions of manic-like symptoms in youth.
Eur Child Adolesc Psychiatry. 2016 Aug;25(8):831-42. doi:
10.1007/s00787-015-0800-7. Epub 2015 Dec 9. PMID: 26650482; PMCID:
PMC4967092.

Vulser H, Lemaitre H, Artiges E, Miranda R, Penttilä J, Struve M, Fadai
T, Kappel V, Grimmer Y, Goodman R, Stringaris A, Poustka L, Conrod P,
Frouin V, Banaschewski T, Barker GJ, Bokde AL, Bromberg U, Büchel C,
Flor H, Gallinat J, Garavan H, Gowland P, Heinz A, Ittermann B, Lawrence
C, Loth E, Mann K, Nees F, Paus T, Pausova Z, Rietschel M, Robbins TW,
Smolka MN, Schumann G, Martinot JL, Paillère-Martinot ML; IMAGEN
Consortium (www.imagen-europe.com); IMAGEN Consortium www imagen-europe
com. Subthreshold depression and regional brain volumes in young
community adolescents. J Am Acad Child Adolesc Psychiatry. 2015
Oct;54(10):832-40. doi: 10.1016/j.jaac.2015.07.006. Epub 2015 Aug 4.
PMID: 26407493.

Aebi M, van Donkelaar MM, Poelmans G, Buitelaar JK, Sonuga-Barke EJ,
Stringaris A, Consortium I, Faraone SV, Franke B, Steinhausen HC, van
Hulzen KJ. Gene-set and multivariate genome-wide association analysis of
oppositional defiant behavior subtypes in
attention-deficit/hyperactivity disorder. Am J Med Genet B
Neuropsychiatr Genet. 2016 Jul;171(5):573-88. doi: 10.1002/ajmg.b.32346.
Epub 2015 Jul 16. PMID: 26184070; PMCID: PMC4715802.

Mikita N, Mehta MA, Zelaya FO, Stringaris A. Using arterial spin
labeling to examine mood states in youth. Brain Behav. 2015
Jun;5(6):e00339. doi: 10.1002/brb3.339. Epub 2015 Apr 20. PMID:
26085964; PMCID: PMC4467773.

Stringaris A, Vidal-Ribas Belil P, Artiges E, Lemaitre H, Gollier-Briant
F, Wolke S, Vulser H, Miranda R, Penttilä J, Struve M, Fadai T, Kappel
V, Grimmer Y, Goodman R, Poustka L, Conrod P, Cattrell A, Banaschewski
T, Bokde AL, Bromberg U, Büchel C, Flor H, Frouin V, Gallinat J, Garavan
H, Gowland P, Heinz A, Ittermann B, Nees F, Papadopoulos D, Paus T,
Smolka MN, Walter H, Whelan R, Martinot JL, Schumann G,
Paillère-Martinot ML; IMAGEN Consortium. The Brain's Response to Reward
Anticipation and Depression in Adolescence: Dimensionality, Specificity,
and Longitudinal Predictions in a Community-Based Sample. Am J
Psychiatry. 2015 Dec;172(12):1215-23. doi:
10.1176/appi.ajp.2015.14101298. Epub 2015 Jun 18. PMID: 26085042.

Cortese S, Ferrin M, Brandeis D, Buitelaar J, Daley D, Dittmann RW,
Holtmann M, Santosh P, Stevenson J, Stringaris A, Zuddas A, Sonuga-Barke
EJ; European ADHD Guidelines Group (EAGG). Cognitive training for
attention- deficit/hyperactivity disorder: meta-analysis of clinical and
neuropsychological outcomes from randomized controlled trials. J Am Acad
Child Adolesc Psychiatry. 2015 Mar;54(3):164-74. doi:
10.1016/j.jaac.2014.12.010. Epub 2014 Dec 29. Erratum in: J Am Acad
Child Adolesc Psychiatry. 2015 May;54(5):433. PMID: 25721181; PMCID:
PMC4382075.

Whelan YM, Leibenluft E, Stringaris A, Barker ED. Pathways from maternal
depressive symptoms to adolescent depressive symptoms: the unique
contribution of irritability symptoms. J Child Psychol Psychiatry. 2015
Oct;56(10):1092-100. doi: 10.1111/jcpp.12395. Epub 2015 Feb 9. PMID:
25665134; PMCID: PMC4855627.

Mikita N, Hollocks MJ, Papadopoulos AS, Aslani A, Harrison S, Leibenluft
E, Simonoff E, Stringaris A. Irritability in boys with autism spectrum
disorders: an investigation of physiological reactivity. J Child Psychol
Psychiatry. 2015 Oct;56(10):1118-26. doi: 10.1111/jcpp.12382. Epub 2015
Jan 28. PMID: 25626926; PMCID: PMC4737220.

Fernández de la Cruz L, Simonoff E, McGough JJ, Halperin JM, Arnold LE,
Stringaris A. Treatment of children with attention-deficit/hyperactivity
disorder (ADHD) and irritability: results from the multimodal treatment
study of children with ADHD (MTA). J Am Acad Child Adolesc Psychiatry.
2015 Jan;54(1):62-70.e3. doi: 10.1016/j.jaac.2014.10.006. Epub 2014 Oct
18. PMID: 25524791; PMCID: PMC4284308.

Vidal-Ribas P, Stringaris A, Rück C, Serlachius E, Lichtenstein P,
Mataix- Cols D. Are stressful life events causally related to the
severity of obsessive- compulsive symptoms? A monozygotic twin
difference study. Eur Psychiatry. 2015 Feb;30(2):309-16. doi:
10.1016/j.eurpsy.2014.11.008. Epub 2014 Dec 12. Erratum in: Eur
Psychiatry. 2015 Jul;30(5):664. PMID: 25511316; PMCID: PMC4331096.

Deveney CM, Hommer RE, Reeves E, Stringaris A, Hinton KE, Haring CT,
Vidal- Ribas P, Towbin K, Brotman MA, Leibenluft E. A prospective study
of severe irritability in youths: 2- and 4-year follow-up. Depress
Anxiety. 2015 May;32(5):364-72. doi: 10.1002/da.22336. Epub 2014 Dec 12.
PMID: 25504765.

Stringaris A, Youngstrom E. In reply. J Am Acad Child Adolesc
Psychiatry. 2014 Nov;53(11):1235-6. doi: 10.1016/j.jaac.2014.08.007.
Epub 2014 Oct 23. PMID: 25440314.

Wiggins JL, Mitchell C, Stringaris A, Leibenluft E. Developmental
trajectories of irritability and bidirectional associations with
maternal depression. J Am Acad Child Adolesc Psychiatry. 2014
Nov;53(11):1191-205, 1205.e1-4. doi: 10.1016/j.jaac.2014.08.005. Epub
2014 Sep 3. PMID: 25440309; PMCID: PMC4254549.

Vidal-Ribas P, Goodman R, Stringaris A. Positive attributes in children
and reduced risk of future psychopathology. Br J Psychiatry. 2015
Jan;206(1):17-25. doi: 10.1192/bjp.bp.114.144519. Epub 2014 Oct 30.
PMID: 25359925; PMCID: PMC4283589.

Stringaris A. Editorial: Trials and tribulations in child psychology and
psychiatry: what is needed for evidence-based practice. J Child Psychol
Psychiatry. 2014 Nov;55(11):1185-6. doi: 10.1111/jcpp.12343. PMID:
25306851.

Daley D, van der Oord S, Ferrin M, Danckaerts M, Doepfner M, Cortese S,
Sonuga-Barke EJ; European ADHD Guidelines Group. Behavioral
interventions in attention-deficit/hyperactivity disorder: a
meta-analysis of randomized controlled trials across multiple outcome
domains. J Am Acad Child Adolesc Psychiatry. 2014 Aug;53(8):835-47,
847.e1-5. doi: 10.1016/j.jaac.2014.05.013. Epub 2014 Jun 26. PMID:
25062591.

Kyriakopoulos M, Stringaris A, Manolesou S, Radobuljac MD, Jacobs B,
Reichenberg A, Stahl D, Simonoff E, Frangou S. Determination of
psychosis- related clinical profiles in children with autism spectrum
disorders using latent class analysis. Eur Child Adolesc Psychiatry.
2015 Mar;24(3):301-7. doi: 10.1007/s00787-014-0576-1. Epub 2014 Jun 26.
PMID: 24965798; PMCID: PMC4224587.

Stringaris A, Castellanos-Ryan N, Banaschewski T, Barker GJ, Bokde AL,
Bromberg U, Büchel C, Fauth-Bühler M, Flor H, Frouin V, Gallinat J,
Garavan H, Gowland P, Heinz A, Itterman B, Lawrence C, Nees F,
Paillere-Martinot ML, Paus T, Pausova Z, Rietschel M, Smolka MN,
Schumann G, Goodman R, Conrod P; Imagen Consortium. Dimensions of manic
symptoms in youth: psychosocial impairment and cognitive performance in
the IMAGEN sample. J Child Psychol Psychiatry. 2014 Dec;55(12):1380-9.
doi: 10.1111/jcpp.12255. Epub 2014 May 28. PMID: 24865127; PMCID:
PMC4167034.

Pan PM, Salum GA, Gadelha A, Moriyama T, Cogo-Moreira H, Graeff-Martins
AS, Rosario MC, Polanczyk GV, Brietzke E, Rohde LA, Stringaris A,
Goodman R, Leibenluft E, Bressan RA. Manic symptoms in youth:
dimensions, latent classes, and associations with parental
psychopathology. J Am Acad Child Adolesc Psychiatry. 2014
Jun;53(6):625-634.e2. doi: 10.1016/j.jaac.2014.03.003. Epub 2014 Mar 22.
PMID: 24839881; PMCID: PMC4477846.

Stringaris A, Youngstrom E. Unpacking the differences in US/UK rates of
clinical diagnoses of early-onset bipolar disorder. J Am Acad Child
Adolesc Psychiatry. 2014 Jun;53(6):609-11. doi:
10.1016/j.jaac.2014.02.013. PMID: 24839878.

Stringaris A, Lewis G, Maughan B. Developmental pathways from childhood
conduct problems to early adult depression: findings from the ALSPAC
cohort. Br J Psychiatry. 2014 Jul;205(1):17-23. doi:
10.1192/bjp.bp.113.134221. Epub 2014 Apr 24. PMID: 24764545; PMCID:
PMC4076653.

Shaw P, Stringaris A, Nigg J, Leibenluft E. Emotion dysregulation in
attention deficit hyperactivity disorder. Am J Psychiatry. 2014
Mar;171(3):276-93. doi: 10.1176/appi.ajp.2013.13070966. PMID: 24480998;
PMCID: PMC4282137.

Dougherty LR, Smith VC, Bufferd SJ, Carlson GA, Stringaris A, Leibenluft
E, Klein DN. DSM-5 disruptive mood dysregulation disorder: correlates
and predictors in young children. Psychol Med. 2014 Aug;44(11):2339-50.
doi: 10.1017/S0033291713003115. Epub 2014 Jan 21. PMID: 24443797; PMCID:
PMC4480202.

Stringaris A. Editorial: mood disorders in families: ways to discovery.
J Child Psychol Psychiatry. 2014;55(2):97-8. doi: 10.1111/jcpp.12203.
Erratum in: J Child Psychol Psychiatry. 2014 Mar;55(3):E1. PMID:
24428689.

Dougherty LR, Smith VC, Bufferd SJ, Stringaris A, Leibenluft E, Carlson
GA, Klein DN. Preschool irritability: longitudinal associations with
psychiatric disorders at age 6 and parental psychopathology. J Am Acad
Child Adolesc Psychiatry. 2013 Dec;52(12):1304-13. doi:
10.1016/j.jaac.2013.09.007. Epub 2013 Sep 26. PMID: 24290463; PMCID:
PMC3860177.

Krieger FV, Leibenluft E, Stringaris A, Polanczyk GV. Irritability in
children and adolescents: past concepts, current debates, and future
opportunities. Braz J Psychiatry. 2013;35 Suppl 1(0 1):S32-9. doi:
10.1590/1516-4446-2013-S107. PMID: 24142126; PMCID: PMC4470558.

Stringaris A. Commentary: bipolar disorder in children and adolescents -
good to have the evidence. Child Adolesc Ment Health. 2013
Sep;18(3):149-150. doi: 10.1111/camh.12036. Epub 2013 Jul 12. PMID:
32847250.

Whelan YM, Stringaris A, Maughan B, Barker ED. Developmental continuity
of oppositional defiant disorder subdimensions at ages 8, 10, and 13
years and their distinct psychiatric outcomes at age 16 years. J Am Acad
Child Adolesc Psychiatry. 2013 Sep;52(9):961-9. doi:
10.1016/j.jaac.2013.06.013. Epub 2013 Aug 1. PMID: 23972698; PMCID:
PMC4026040.

Bolhuis K, McAdams TA, Monzani B, Gregory AM, Mataix-Cols D, Stringaris
A, Eley TC. Aetiological overlap between obsessive-compulsive and
depressive symptoms: a longitudinal twin study in adolescents and
adults. Psychol Med. 2014 May;44(7):1439-49. doi:
10.1017/S0033291713001591. Epub 2013 Aug 7. PMID: 23920118; PMCID:
PMC3959155.

Stringaris A, Maughan B, Copeland WS, Costello EJ, Angold A. Irritable
mood as a symptom of depression in youth: prevalence, developmental, and
clinical correlates in the Great Smoky Mountains Study. J Am Acad Child
Adolesc Psychiatry. 2013 Aug;52(8):831-40. doi:
10.1016/j.jaac.2013.05.017. Epub 2013 Jul 3. PMID: 23880493; PMCID:
PMC3728563.

Stoddard J, Stringaris A, Brotman MA, Montville D, Pine DS, Leibenluft
E. Irritability in child and adolescent anxiety disorders. Depress
Anxiety. 2014 Jul;31(7):566-73. doi: 10.1002/da.22151. Epub 2013 Jul 1.
PMID: 23818321; PMCID: PMC3937265.

Krieger FV, Stringaris A. Bipolar disorder and disruptive mood
dysregulation in children and adolescents: assessment, diagnosis and
treatment. Evid Based Ment Health. 2013 Nov;16(4):93-4. doi:
10.1136/eb-2013-101400. Epub 2013 Jun 8. PMID: 23749629.

Stringaris A. Here/in this issue and there/abstract thinking: gene
effects cross the boundaries of psychiatric disorders. J Am Acad Child
Adolesc Psychiatry. 2013 Jun;52(6):557-8. doi:
10.1016/j.jaac.2013.03.015. PMID: 23702441.

Stringaris A, Goodman R. The value of measuring impact alongside
symptoms in children and adolescents: a longitudinal assessment in a
community sample. J Abnorm Child Psychol. 2013 Oct;41(7):1109-20. doi:
10.1007/s10802-013-9744-x. PMID: 23677767; PMCID: PMC3755220.

Stringaris A. Editorial: The new DSM is coming--it needs tough
love\ldots{} J Child Psychol Psychiatry. 2013 May;54(5):501-2. doi:
10.1111/jcpp.12078. PMID: 23662786; PMCID: PMC3736248.

Krieger FV, Polanczyk VG, Goodman R, Rohde LA, Graeff-Martins AS, Salum
G, Gadelha A, Pan P, Stahl D, Stringaris A. Dimensions of
oppositionality in a Brazilian community sample: testing the DSM-5
proposal and etiological links. J Am Acad Child Adolesc Psychiatry. 2013
Apr;52(4):389-400.e1. doi: 10.1016/j.jaac.2013.01.004. PMID: 23582870;
PMCID: PMC3834546.

Maughan B, Collishaw S, Stringaris A. Depression in childhood and
adolescence. J Can Acad Child Adolesc Psychiatry. 2013 Feb;22(1):35-40.
PMID: 23390431; PMCID: PMC3565713.

Cortese S, Holtmann M, Banaschewski T, Buitelaar J, Coghill D,
Danckaerts M, Dittmann RW, Graham J, Taylor E, Sergeant J; European ADHD
Guidelines Group. Practitioner review: current best practice in the
management of adverse events during treatment with ADHD medications in
children and adolescents. J Child Psychol Psychiatry. 2013
Mar;54(3):227-46. doi: 10.1111/jcpp.12036. Epub 2013 Jan 7. PMID:
23294014.

DeSousa DA, Stringaris A, Leibenluft E, Koller SH, Manfro GG, Salum GA.
Cross-cultural adaptation and preliminary psychometric properties of the
Affective Reactivity Index in Brazilian Youth: implications for DSM-5
measured irritability. Trends Psychiatry Psychother. 2013;35(3):171-80.
doi: 10.1590/s2237-60892013000300004. PMID: 25923389.

Mikita N, Stringaris A. Mood dysregulation. Eur Child Adolesc
Psychiatry. 2013 Feb;22 Suppl 1(Suppl 1):S11-6. doi:
10.1007/s00787-012-0355-9. PMID: 23229139; PMCID: PMC3560944.

Stringaris A. Predicting treatment outcomes: encouraging findings from
neuroimaging. J Am Acad Child Adolesc Psychiatry. 2012
Dec;51(12):1227-8. doi: 10.1016/j.jaac.2012.09.019. PMID: 23200277.

Stringaris A, Rowe R, Maughan B. Mood dysregulation across developmental
psychopathology--general concepts and disorder specific expressions. J
Child Psychol Psychiatry. 2012 Nov;53(11):1095-7. doi:
10.1111/jcpp.12003. PMID: 23061783.

Krebs G, Bolhuis K, Heyman I, Mataix-Cols D, Turner C, Stringaris A.
Temper outbursts in paediatric obsessive-compulsive disorder and their
association with depressed mood and treatment outcome. J Child Psychol
Psychiatry. 2013 Mar;54(3):313-22. doi:
10.1111/j.1469-7610.2012.02605.x. Epub 2012 Sep 8. PMID: 22957831;
PMCID: PMC4026039.

Stringaris A. What we can all learn from the Treatment of Early Age
Mania (TEAM) trial. J Am Acad Child Adolesc Psychiatry. 2012
Sep;51(9):861-3. doi: 10.1016/j.jaac.2012.06.014. PMID: 22917198.

Aebi M, Kuhn C, Metzke CW, Stringaris A, Goodman R, Steinhausen HC. The
use of the development and well-being assessment (DAWBA) in clinical
practice: a randomized trial. Eur Child Adolesc Psychiatry. 2012
Oct;21(10):559-67. doi: 10.1007/s00787-012-0293-6. Epub 2012 Jun 22.
PMID: 22722664; PMCID: PMC3866649.

Stringaris A. In this issue/abstract thinking: treatment response in
psychiatry. J Am Acad Child Adolesc Psychiatry. 2012 Jun;51(6):561-2.
doi: 10.1016/j.jaac.2012.03.014. PMID: 22632613; PMCID: PMC3401368.

Weathers JD, Stringaris A, Deveney CM, Brotman MA, Zarate CA Jr,
Connolly ME, Fromm SJ, LeBourdais SB, Pine DS, Leibenluft E. A
developmental study of the neural circuitry mediating motor inhibition
in bipolar disorder. Am J Psychiatry. 2012 Jun;169(6):633-41. doi:
10.1176/appi.ajp.2012.11081244. PMID: 22581312; PMCID: PMC3466815.

Stringaris A, Goodman R, Ferdinando S, Razdan V, Muhrer E, Leibenluft E,
Brotman MA. The Affective Reactivity Index: a concise irritability scale
for clinical and research settings. J Child Psychol Psychiatry. 2012
Nov;53(11):1109-17. doi: 10.1111/j.1469-7610.2012.02561.x. Epub 2012 May
10. PMID: 22574736; PMCID: PMC3484687.

Leigh E, Smith P, Milavic G, Stringaris A. Mood regulation in youth:
research findings and clinical approaches to irritability and
short-lived episodes of mania-like symptoms. Curr Opin Psychiatry. 2012
Jul;25(4):271-6. doi: 10.1097/YCO.0b013e3283534982. PMID: 22569307;
PMCID: PMC3660700.

Moyá J, Stringaris AK, Asherson P, Sandberg S, Taylor E. The impact of
persisting hyperactivity on social relationships: a community-based,
controlled 20-year follow-up study. J Atten Disord. 2014
Jan;18(1):52-60. doi: 10.1177/1087054712436876. Epub 2012 Mar 21. PMID:
22441888; PMCID: PMC3867339.

Stringaris A, Zavos H, Leibenluft E, Maughan B, Eley TC. Adolescent
irritability: phenotypic associations and genetic links with depressed
mood. Am J Psychiatry. 2012 Jan;169(1):47-54. doi:
10.1176/appi.ajp.2011.10101549. Epub 2011 Oct 31. Erratum in: Am J
Psychiatry. 2013 Jul 1;170(7):810. PMID: 22193524; PMCID: PMC3660701.

Stringaris A. In this issue/abstract thinking: clinical diagnoses and
the future of biomarkers. J Am Acad Child Adolesc Psychiatry. 2011
Dec;50(12):1197-8. doi: 10.1016/j.jaac.2011.09.011. PMID: 22115137.

Stringaris A, Stahl D, Santosh P, Goodman R. Dimensions and latent
classes of episodic mania-like symptoms in youth: an empirical enquiry.
J Abnorm Child Psychol. 2011 Oct;39(7):925-37. doi:
10.1007/s10802-011-9520-8. PMID: 21625986; PMCID: PMC3161193.

Chan J, Stringaris A, Ford T. Bipolar Disorder in Children and
Adolescents Recognised in the UK: A Clinic-Based Study. Child Adolesc
Ment Health. 2011 May;16(2):71-78. doi:
10.1111/j.1475-3588.2010.00566.x. PMID: 32847219.

Stringaris A. Irritability in children and adolescents: a challenge for
DSM-5. Eur Child Adolesc Psychiatry. 2011 Feb;20(2):61-6. doi:
10.1007/s00787-010-0150-4. Epub 2011 Feb 5. PMID: 21298306.

Stringaris A. Abstract thinking: environmental modification,
development, and psychopathology. J Am Acad Child Adolesc Psychiatry.
2010 Dec;49(12):1180. doi: 10.1016/j.jaac.2010.09.010. PMID: 21093765.

Stringaris A, Maughan B, Goodman R. What's in a disruptive disorder?
Temperamental antecedents of oppositional defiant disorder: findings
from the Avon longitudinal study. J Am Acad Child Adolesc Psychiatry.
2010 May;49(5):474-83. doi: 10.1097/00004583-201005000-00008. PMID:
20431467.

Stringaris A, Baroni A, Haimm C, Brotman M, Lowe CH, Myers F, Rustgi E,
Wheeler W, Kayser R, Towbin K, Leibenluft E. Pediatric bipolar disorder
versus severe mood dysregulation: risk for manic episodes on follow-up.
J Am Acad Child Adolesc Psychiatry. 2010 Apr;49(4):397-405. PMID:
20410732; PMCID: PMC3000433.

Sobanski E, Banaschewski T, Asherson P, Buitelaar J, Chen W, Franke B,
Holtmann M, Krumm B, Sergeant J, Sonuga-Barke E, Stringaris A, Taylor E,
Anney R, Ebstein RP, Gill M, Miranda A, Mulas F, Oades RD, Roeyers H,
Rothenberger A, Steinhausen HC, Faraone SV. Emotional lability in
children and adolescents with attention deficit/hyperactivity disorder
(ADHD): clinical correlates and familial prevalence. J Child Psychol
Psychiatry. 2010 Aug;51(8):915-23. doi:
10.1111/j.1469-7610.2010.02217.x. Epub 2010 Feb 1. PMID: 20132417.

Stringaris A, Santosh P, Leibenluft E, Goodman R. Youth meeting symptom
and impairment criteria for mania-like episodes lasting less than four
days: an epidemiological enquiry. J Child Psychol Psychiatry. 2010
Jan;51(1):31-8. doi: 10.1111/j.1469-7610.2009.02129.x. Epub 2009 Jul 22.
PMID: 19686330; PMCID: PMC4286871.

Stringaris A, Cohen P, Pine DS, Leibenluft E. Adult outcomes of youth
irritability: a 20-year prospective community-based study. Am J
Psychiatry. 2009 Sep;166(9):1048-54. doi:
10.1176/appi.ajp.2009.08121849. Epub 2009 Jul 1. PMID: 19570932; PMCID:
PMC2791884.

Stringaris A, Goodman R. Longitudinal outcome of youth oppositionality:
irritable, headstrong, and hurtful behaviors have distinctive
predictions. J Am Acad Child Adolesc Psychiatry. 2009 Apr;48(4):404-412.
doi: 10.1097/CHI.0b013e3181984f30. PMID: 19318881.

Stringaris A, Goodman R. Three dimensions of oppositionality in youth. J
Child Psychol Psychiatry. 2009 Mar;50(3):216-23. doi:
10.1111/j.1469-7610.2008.01989.x. Epub 2008 Oct 23. PMID: 19166573.

Stringaris A, Goodman R. Mood lability and psychopathology in youth.
Psychol Med. 2009 Aug;39(8):1237-45. doi: 10.1017/S0033291708004662.
Epub 2008 Dec 11. PMID: 19079807.

Schmidt H, Stuertz K, Chen V, Stringaris AK, Brück W, Nau R. Glycerol
does not reduce neuronal damage in experimental Streptococcus pneumoniae
meningitis in rabbits. Inflammopharmacology. 1998;6(1):19-26. doi:
10.1007/s10787-998-0003-7. PMID: 17638124.

Stringaris AK, Medford N, Giora R, Giampietro VC, Brammer MJ, David AS.
How metaphors influence semantic relatedness judgments: the role of the
right frontal cortex. Neuroimage. 2006 Nov 1;33(2):784-93. doi:
10.1016/j.neuroimage.2006.06.057. Epub 2006 Sep 11. PMID: 16963282.

Stringaris AK, Medford NC, Giampietro V, Brammer MJ, David AS. Deriving
meaning: Distinct neural mechanisms for metaphoric, literal, and
non-meaningful sentences. Brain Lang. 2007 Feb;100(2):150-62. doi:
10.1016/j.bandl.2005.08.001. Epub 2005 Sep 13. PMID: 16165201.

Iliev AI, Stringaris AK, Nau R, Neumann H. Neuronal injury mediated via
stimulation of microglial toll-like receptor-9 (TLR9). FASEB J. 2004
Feb;18(2):412-4. doi: 10.1096/fj.03-0670fje. Epub 2003 Dec 19. PMID:
14688201.

Stringaris AK, Geisenhainer J, Bergmann F, Balshüsemann C, Lee U, Zysk
G, Mitchell TJ, Keller BU, Kuhnt U, Gerber J, Spreer A, Bähr M, Michel
U, Nau R. Neurotoxicity of pneumolysin, a major pneumococcal virulence
factor, involves calcium influx and depends on activation of p38
mitogen-activated protein kinase. Neurobiol Dis. 2002 Dec;11(3):355-68.
doi: 10.1006/nbdi.2002.0561. PMID: 12586546.

Michel U, Ebert S, Schneider O, Shintani Y, Bunkowski S, Smirnov A,
Stringaris A, Gerber J, Brück W, Nau R. Follistatin (FS) in human
cerebrospinal fluid and regulation of FS expression in a mouse model of
meningitis. Eur J Endocrinol. 2000 Dec;143(6):809-16. doi:
10.1530/eje.0.1430809. PMID: 11124865.

Michel U, Stringaris AK, Nau R, Rieckmann P. Differential expression of
sense and antisense transcripts of the mitochondrial DNA region coding
for ATPase 6 in fetal and adult porcine brain: identification of novel
unusually assembled mitochondrial RNAs. Biochem Biophys Res Commun. 2000
Apr 29;271(1):170-80. doi: 10.1006/bbrc.2000.2595. PMID: 10777698.

Bitsch A, Bruhn H, Vougioukas V, Stringaris A, Lassmann H, Frahm J,
Brück W. Inflammatory CNS demyelination: histopathologic correlation
with in vivo quantitative proton MR spectroscopy. AJNR Am J Neuroradiol.
1999 Oct;20(9):1619-27. PMID: 10543631; PMCID: PMC7056180.

Schmidt H, Stuertz K, Brück W, Chen V, Stringaris AK, Fischer FR, Nau R.
Intravenous granulocyte colony-stimulating factor increases the release
of tumour necrosis factor and interleukin-1beta into the cerebrospinal
fluid, but does not inhibit the growth of Streptococcus pneumoniae in
experimental meningitis. Scand J Immunol. 1999 May;49(5):481-6. doi:
10.1046/j.1365-3083.1999.00518.x. PMID: 10320640.

Schmidt H, Zysk G, Reinert RR, Brück W, Stringaris A, Chen V, Stuertz K,
Fischer F, Bartels R, Schaper KJ, Weinig S, Nau R. Rifabutin for
experimental pneumococcal meningitis. Chemotherapy. 1997
Jul-Aug;43(4):264-71. doi: 10.1159/000239577. PMID: 9209783.

Nau R, Zysk G, Schmidt H, Fischer FR, Stringaris AK, Stuertz K, Brück W.
Trovafloxacin delays the antibiotic-induced inflammatory response in
experimental pneumococcal meningitis. J Antimicrob Chemother. 1997
Jun;39(6):781-8. doi: 10.1093/jac/39.6.781. PMID: 9222048.

Stringaris AK, Brück W, Tumani H, Schmidt H, Nau R. Increased glutamine
synthetase immunoreactivity in experimental pneumococcal meningitis.
Acta Neuropathol. 1997 Mar;93(3):215-8. doi: 10.1007/s004010050606.
PMID: 9083551.

\end{document}
